% Created 2021-10-02 Sat 16:38
% Intended LaTeX compiler: pdflatex
\documentclass[11pt]{article}
\usepackage[utf8]{inputenc}
\usepackage[T1]{fontenc}
\usepackage{graphicx}
\usepackage{grffile}
\usepackage{longtable}
\usepackage{wrapfig}
\usepackage{rotating}
\usepackage[normalem]{ulem}
\usepackage{amsmath}
\usepackage{textcomp}
\usepackage{amssymb}
\usepackage{capt-of}
\usepackage{hyperref}
\usepackage{minted}
\author{NTNU IDATA 2302}
\date{Sep. 29, 2021}
\title{Sorting\\\medskip
\large Week 6}
\hypersetup{
 pdfauthor={NTNU IDATA 2302},
 pdftitle={Sorting},
 pdfkeywords={},
 pdfsubject={},
 pdfcreator={Emacs 27.2 (Org mode 9.4.4)}, 
 pdflang={English}}
\begin{document}

\maketitle


\section{Questions}
\label{sec:org63a206b}


\begin{enumerate}
\item What is worst case runtime efficiency of the "quick sort"
algorithm?
\begin{enumerate}
\item \(O(n^3)\)
\item \(O(n^2)\)
\item \(O(n \log n)\)
\item \(O(n)\)
\item \(O(\log n)\)
\end{enumerate}

\item In the \emph{quick sort}, how would you \textbf{not} choose the pivot?
\begin{enumerate}
\item The middle element in the array
\item The last element
\item The first element
\item The maximum element
\item The minimum element
\end{enumerate}

\item In \emph{quick sort}, what is the runtime efficiency of partitioning an
array?
\begin{enumerate}
\item \(O(n^3)\)
\item \(O(n^2)\)
\item \(O(n \log n)\)
\item \(O(n)\)
\item \(O(\log n)\)
\end{enumerate}

\item What is the best case runtime efficiency of the merge sort?
\begin{enumerate}
\item \(O(n^3)\)
\item \(O(n^2)\)
\item \(O(n \log n)\)
\item \(O(n)\)
\item \(O(\log n)\)
\end{enumerate}

\item In the worst case, \emph{merge sort} runs theoretically faster than
quick sort?
\begin{enumerate}
\item Yes
\item No
\item One cannot say
\end{enumerate}

\item What is the correct recursive call of \emph{merge sort}?
\begin{enumerate}
\item \texttt{split(merge(sort(array))}
\item \texttt{merge(sort(left), sort(right))}
\item \texttt{sort(merge(left), right(right))}
\item None of the above
\end{enumerate}

\item Comparison-based sorting algorithms are bound by \(n \log n\). What
kind of scenario does it relate to?
\begin{enumerate}
\item Worst case
\item Average case
\item Best case
\end{enumerate}

\item Comparison-based sorting algorithms are bound by \(n \log n\). What
kind of bound is this?
\begin{enumerate}
\item An upper bound
\item A lower bound
\item Both a lower and an upper bound.
\end{enumerate}

\item The runtime complexity of the counting sort is \(O(k + n)\) where \(n\)
denotes the length of the given array. What does \(k\) stands for?
\begin{enumerate}
\item The capacity of the given array
\item The maximum element in the array
\item The number of symbol in the underlying alphabet
\item None of the above
\end{enumerate}

\item The runtime complexity of the \emph{radix sort} is \(O(d \cdot
    (n+k))\), where \(n\) is the length of the array. What does \(d\)
represent?
\begin{enumerate}
\item The size of the symbol alphabet
\item The maximum symbol
\item The maximum number of digits
\item The distance between to symbol
\item None of the above
\end{enumerate}
\end{enumerate}


\section{Solutions}
\label{sec:orgee56959}

\begin{enumerate}
\item \textbf{Answer (b).} In the worst case, \emph{quick sort} runs in \(O(n^2)\).

\item \textbf{Answers (d) and (e)}. Choosing the maximum or the minimum
precisely lead to the worst runtime for quick sort.

\item \textbf{Answer (d)} The partitioning used in quick sort runs in \(O(n)\).

\item \textbf{Answer (c)} In all cases, merge sort runs in \(O(n \log n)\).

\item \textbf{Answer (a)} Yes, in the worst case, merge sort runs in \(O(n \log
     n)\) whereas quick sort runs in \(O(n^2)\).

\item \textbf{Answer (b)}. \texttt{merge(sort(left), sort(right))} is the correct
recursive call. We first split, then sort both sub-arrays using
merge sort, and then merge the results.

\item \textbf{Answer (a)}. It relates to the worst case.

\item \textbf{Answer (b)}. It relates to the lower bound.

\item \textbf{Answer (c)}. The number of symbols in the underlying alphabet.

\item \textbf{Answer (c)}. The maximum number of digits in the numbers to
sort.
\end{enumerate}
\end{document}