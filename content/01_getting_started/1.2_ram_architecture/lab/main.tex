\documentclass{aldast}


\title{RAM Programming}
\author{F. Chauvel}
\documentType{Lab Session}
\documentNumber{1.2}

\begin{document}

\maketitle

We will look at the RAM model and we will write some tiny
programs. For this we only need pen-and-paper. The point is not to
memorize the RAM instructions, but to anchor the concepts and to
understand how a machine works. This is necessary to later understand
why some algorithms are more efficient than others. For every RAM
program, I make the following assumptions:
\begin{itemize}
\item The code is stored from address 0. I put the data further down
  into the memory.
\item I don't know what is in the register or the memory beforehand.
\end{itemize}

\begin{exercise}[{What Does This RAM Program Do?}]
Below is a three lines RAM program. What does this program do?
\begin{minted}{asm}
  LOAD 0
  ADD 25
  STORE 35
\end{minted}
\end{exercise}

This program copies the content of the memory cell at address 25 into
the cell at address 35. The first \texttt{LOAD} instruction set the
\texttt{ACC} register to 0. The second instruction copy the value
contained in memory at address 25 into the \texttt{ACC}
register. Finally the last instruction copy the content of the
\texttt{ACC} register into the memory, at address 35.

\begin{exercise}[{Swapping two memory cells}]
  Write a RAM program that swaps the content two memory cells, say 100
  and 101. For example, if address 100 contains 25 and address 101
  contains 50, after your program executes, address 100 must contains
  50 and address 101 must contain 25.
\end{exercise}

Swapping two memory cells requires to use a third one, say 102 for
example. I don't see any other way with the constraint of the RAM. We
first copy the first cell (with address 100) into another free cell
(say 102) and we then override the content of the first cell (100),
with the content of the second one (101). Finally, we override the
content of the second cell, with the value we saved in the third cell
(102). The code below shows the sequences of RAM instructions.

\begin{minted}{asm}
  LOAD 0        ;; Copy Cell 100 into Cell 102
  ADD 100
  STORE 102
  LOAD 0        ;; Copy Cell 101 into Cell 100
  ADD 101
  STORE 100
  LOAD 0        ;; Copy Cell 102 into Cell 101 
  ADD 102
  STORE 101
\end{minted}


\begin{exercise}[{Maximum of Two Numbers}]
  Write a RAM program that reads two numbers from the I/O device and
  prints the largest one.
\end{exercise}

In the following program, we first read two values from the I/O
device, which we store at address 100 and 101, respectively. We then
load the first one into the \texttt{ACC} register and subtract the
other one. The \texttt{ACC} thus contains the difference. We can then
\texttt{JUMP} (only if this difference is positive) to Line 10 where
we print the first value. Otherwise, we proceed and print the second
value, and jump to Line 11 (because we set the \texttt{ACC} to zero
previously).

\begin{minted}{asm}
  READ 100      ;; Read two numbers 
  READ 101
  LOAD 0        ;; Compute the difference
  ADD 100
  SUB 101
  JUMP 18       ;; Jump to Line 10, if the difference is positive
  PRINT 100     ;; Print the second value
  LOAD 0
  JUMP 20       ;; Jump to Line 11, always.
  PRINT 101     ;; Print the first value
  HALT
\end{minted}

\begin{exercise}[{Multiplication}]
  Write a RAM program that reads two numbers from the I/O device and
  then computes and prints their product back onto the I/O device.
\end{exercise}

I found this program slightly more involved. Since the RAM model only
gives us an addition instructions, we need to express the
multiplication as follows:

\begin{equation}
  a \times b = \overbrace{a + a + a + \ldots + a}^{\textrm{b times}}
\end{equation}

To multiple two values, say $a$ and $b$, we will encode a loop that
adds $a$ b times to the \texttt{ACC} register and then print the
result. We will therefore need two additional variables:
\texttt{counter} (stored at address 102) which counts how many times
we have added a so far, and \texttt{result} (stored at address 103)
which hold the on going accumulation.

\begin{minted}{asm}
  READ 100    ;; a := I/O
  READ 101    ;; b := I/O
  LOAD 0
  STORE 102   ;; counter := 0
  LOAD 0
  STORE 103   ;; result := 0
  LOAD 0
  ADD 102
  SUB 101     ;; 
  JUMP 38     ;; if counter - b >= 0, go print result (Line 20)
  LOAD 0
  ADD 103       
  ADD 100
  STORE 103   ;; result := result + a
  LOAD 1
  ADD 102
  STORE 102   ;; counter := counter + 1
  LOAD 0
  JUMP 12     ;; go back to the loop condition (Line 7)
  PRINT 103   ;; print result
  HALT
\end{minted}

\end{document}