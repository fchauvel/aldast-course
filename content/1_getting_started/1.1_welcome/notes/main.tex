\documentclass{aldast}


\documentType{Lecture Notes}
\documentNumber{1.1}
\title{Getting Started}
\author{F. Chauvel}


\begin{document}

\maketitle

\begin{abstract}
  Let us first take a look at the course as a whole: What topics will
  we cover and which one will we leave aside. We also look at our
  respective expectations: What do I expect from you and what can you
  reasonably expect from me. I then go through the course
  organizations, with lectures, lab sessions, weekly quizzes and
  examinations. Finally, we look at external resources (books, online
  courses, websites, etc.) which you may find helpful.
\end{abstract}


\section*{Introduction}
Welcome. I\sidenote{My name is Franck. I am a full stack engineer at
  Axbit AS and a lecturer at NTNU. Contact
  \href{mailto:franck.chauvel@ntnu.no}{me} if I can help!} am very
happy to welcome you in this course about \emph{Algorithms \& Data
  Structures}. I prepared this course as well as I could with the
time, resources and energy I had, and I do hope you will find it
relevant. \emph{Feedback is always welcome.}


\section{Algorithms and Data Structures}
% What do we want to learn
I guess you already came across these terms: Algorithms and data
structures. What are these? To start simple, let us take an algorithm
as a cooking recipe, that is, a \emph{ordered} sequence of steps. Data
structures would then resemble the way the kitchen is organized. If all
kitchen equipment and ingredients are readily available, the chef will
be efficient.

Algorithms and data structures encompass very general knowledge about
how to wrote \emph{efficient} software, regardless of the languages,
framework and machines. What we will study is ``rock solid''!

\paragraph{What will we study?} This course starts from scratch: There
is no prerequisite, except maybe some high school calculus
(probabilities, summations and series). Table~\ref{tab:syllabus}
details the topics covered in each of the 13 weeks. We start with the
computer architecture (CPU \& memory) and core programming concepts
(variables, conditional and loops) and then move to common data
structures: Record, arrays, and lists. At the same time we will study
simple algorithms to search and sort.

We will then push on to more advanced concepts, including hashing
(Week 7), trees (Week 8--9) and graphs (Week 10). These are all the
data structure we will study. In the remainder we will explore two
other topics where algorithms play a key role: Discrete optimization
and compilation.

Finally we will wrap up with looking at other computation models,
namely parallel and quantum computing. The knowledge we have gathered
is very general and apply there as well.

\begin{table}[p]
  \begin{center}
  \begin{tabular}{llll}
    \toprule
    Week & N    & Lecture                   & Lab        \\
    \midrule 
    1    & 1.1  & Getting Started           & Java Setup \\
         & 1.2  & Random Access Machine     &            \\
         & 1.3  & Correctness               & TDD        \\
    2    & 2.1  & Efficiency                &            \\
         & 2.2  & Algorithm Analysis        &            \\
         & 2.3  & Order of growth           &            \\
    \midrule
    3    & 3.1  & Static Arrays             &            \\
         & 3.2  & Dynamic Arrays            &            \\
         & 3.3  & Searching                 &            \\
    4    & 4.1  & Sorting                   &            \\
         & 4.2  & Linked Lists              &            \\
         & 4.3  & Stacks and Queues         &            \\
    5    & 5.1  & Procedure Calls           &            \\
         & 5.2  & Recursion                 &            \\
         & 5.3  & Recurrences (optional)    &            \\
    6    & 6.1  & Quick sort                &            \\
         & 6.2  & Merge sort                &            \\
         & 6.3  & Beyond the sorting bound  &            \\
    \midrule
    7    & 7.1  & Hash tables               &            \\
         & 7.2  & Collisions                &            \\
         & 7.3  & Bloom filters             &            \\
    \midrule
    8    & 8.1  & Trees                     &            \\
         & 8.2  & Binary search trees       &            \\
         & 8.3  & Self-balancing trees      &            \\
    9    & 9.1  & Binary Heaps              &            \\
         & 9.2  & Prefix trees              &            \\
         & 9.3  & B-trees                   &            \\
    \midrule
    10   & 10.1 & Graphs                    &            \\
         & 10.2 & Shortest paths            &            \\
         & 10.3 & Spanning trees \& more    &            \\
    \midrule
    11   & 11.1 & Combinatorial Enumeration &            \\
         & 11.2 & Dynamic Programming       &            \\
         & 11.3 & Heuristic Search          &            \\
    \midrule
    12   & 12.1 & String matching           &            \\
         & 12.2 & Regular expressions       &            \\
         & 12.3 & NP-Completeness           &            \\
    \midrule
    13   & 13.1 & Algorithm Design          &            \\
         & 13.2 & Parallel Computing        &            \\
         & 13.3 & Quantum Computing         &            \\
    \bottomrule
  \end{tabular}
  \end{center}
  \caption{Course syllabus over 13 weeks}
  \label{tab:syllabus}
\end{table}

\paragraph{Why to study this?} Algorithms and data structures form
the core of Computer Science, and in turn, of Software
Engineering. Image processing, cryptography, compilers, networks,
artificial intelligence, and other ``branches'' of Computer Science
all develop algorithms and data structures. Consider image processing as
an example: How to detect the contour of a shape in a bitmap?  What data
structure is the most suitable to represent an image in memory? Which
algorithm is the fastest? Which consume the least memory? This course
lay down the foundations to discuss, evaluate and compare algorithms
and data structures.

Besides, as a software engineer, your daily work is to choose among
existing data structures and algorithms, depending on the task at
end. You have to sort the score of the top players in your mobile
game. Shall we go for a \emph{quick sort} or a \emph{radix sort}?
Sometimes, even, we have to roll out our own. It is critical to know what
already exists and where we it shines and where it falls short.

\section{Objectives \& Expectations}

\paragraph{Objectives} The main objective is to give you some
exposure---and ideally some fluency---with concepts that every
programmer must know. This goes beyond being able to choose between an
\texttt{ArrayList}, and \texttt{LinkedList} or a \texttt{HashMap} in
Java. The point is also for you to understand what lays behind these
names and how you can possibly make your own. I see the three
following objectives:

\begin{enumerate}
  \item Understand how the machine executes your algorithms;
  \item Know the basic data structures and their operations;
  \item Understand recursion;
  \item Understand algorithms for searching and sorting;
  \item Know how to estimate the time and memory effectiveness of;
    algorithms;
  \item Get better at solving programming problems.
\end{enumerate}

\paragraph{What do I expect from you?}

\emph{These topics is hard}---I cannot stress this enough. We will
cover many, some of which are non-trivial. Yet, I made everything
optional, but the final examination. I am not there to burden
you. To me, this course is an opportunity to study algorithms and data
structures. Here is what I \emph{strongly} recommend:

\begin{itemize}
\item Attend the lecture physically. Even more, ask questions!
  Attending empower you to ask questions and to stop me if I am
  unclear. Please do so: I am happy to adjust or answer to the best of
  my knowledge.
\item Attend the lab sessions to go through the exercises. In my
  experience, exercises (whether on paper or on computer) is the only
  thing that really ground understanding.
\item Go through the home examinations. I believe this is a good
  preparation to the final one.
\item Reach out if you need any help.
\item Practice. practice, practice.
\end{itemize}

As you can see in Table~\ref{tab:syllabus}, this course is
``content-heavy''. I don't think one can master the whole thing in a
week or so before the final examination. A strongly recommend you
spread your work over the 13 weeks.

\paragraph {What can you expect from me?}

I am there three times a week for all lectures and lab sessions. Feel
free to come and ask questions. Besides these slots, I try to be
available and responsive for any one who reach out, by email
especially\sidenote{I am not sitting in NTNU on a daily
  basis. Reaching out by email is the best way get help.}. I try to
reply within 24 hours, but bear in mind that I am also a full time
engineer. Expect answer outside office hours.

\section{Practicalities}
Overall, we have three sessions a week during 13 weeks. Each session is
divided into 30 minutes of lectures and 1h30 of exercises or lab
session.

\paragraph {Lectures}
I broke down the content into 30 minutes long lectures. Each lecture
address a specific topic. I use various programming languages as
examples, avoiding those that are too verbose on slides. Keep in mind
that algorithms and data structures apply to any programming
language. I provide written lecture notes for the most important
lectures, so you can focus on understanding.

\paragraph {Lab sessions}
The lab sessions aim at putting concepts into practice. These are
either pen-and-paper exercises or programming tasks. Programming tasks
are in Java, but feel free to use any other language that works better
for you. The point is to solve the programming tasks, not to struggle
with the language. There is however no need to know Java programming
beforehand, but prior exposure will help. I provide selected written
solution, where I see fit.

\paragraph {Weekly Quizzes}
Every week, I will post a quiz to help you assess your understanding
of the week's topics. They do not account in your grade but please
reach out if any thing is unclear.

\paragraph{Examinations}
Your grade depends solely on the final examination. To get prepared as
well as possible, I include two ``home'' examinations in Week 4 and
8. These are optional, but should you feel the need for it, I can
grade them. All examinations follow the same pattern: 100 points
divided among four parts with 5 questions.

\section{Additional Resources}

There are many many books
\cite{atto1974,melhorn2008,levitin2011,weiss2014,skiena2020}, videos, online
courses and tutorials on algorithms and data structures. Let me know
if my presentation gets unclear or does not fits your taste: We can
look at other resources where the same content is taken from a
different angle.

\emph{I do not follow any textbook} in particular, but there are a few
``references'' textbooks, whose authors are leading authorities in the
field. Feel free to check them if you need a different approach:

\begin{itemize}
\item ``The Art of Computer Programming'' (aka TAOCP) by Donald
  E. Knuth. This is \emph{the} reference. Five volumes so far:
  Fundamentals~\cite{knuth1978}, semi-numerical~\cite{knuth1997},
  sorting and searching~\cite{knuth1998}, combinato\-rial
  al\-go\-ri\-thms~\cite{knuth2011}. I found the treatment very detailed
  and oriented towards the mathematically oriented. Beyond the scope
  of this course.
\item ``Introduction to Algorithms'' by Thomas H. Cormen et
  al~\cite{cormen2009} is another famous reference. It covers most of
  the topic I selected and many more, except maybe random access
  machines.
\item ``Algorithms'' by Robert Sedgewick~\cite{sedgewick2014} is another
  well-known textbook, which is used in some very successful online
  courses. Example are given in Java.
\item ``Data structures and algorithms in Java'' by Michael
  T. Goodrich and Roberto Tamassia~\cite{goodrich2014} fits very well this
  course. Most content we will cover is also treated there.
\end{itemize}

There are also online course on platforms such as
\href{https://www.coursera.org}{Coursera} or
\href{https://www.edx.org}{EdX} for instance. They can be a good
alternative to my presentations. Check however that the content they
cover matches the syllabus.

You will also find many online explanation and tutorials such as
\href{www.geeksforgeeks.com}{GeeksForGeeks} or
\href{www.tutorialspoint.com}{TutorialsPoint}. Please keep a critical
eye as they may contain mistakes.

Finally, there are many web sites such
\href{http://www.hackerrank.com}{HackerRank} or
\href{https://leetcode.com}{LeetCode} where one can practice
programming. Many problems we find there are direct application of the
concepts we will study. These web sites are great resources to gain
fluency.


\section*{Conclusions}
Let us get started! There is a lot or ground to cover. Please reach
out if you have any questions or if you find mistakes in the slides,
the lectures notes or the lab sessions.nn

\bibliographystyle{acm}
\bibliography{references}

\end{document}
