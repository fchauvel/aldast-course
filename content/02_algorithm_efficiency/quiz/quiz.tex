% Created 2021-09-19 Sun 08:02
% Intended LaTeX compiler: pdflatex
\documentclass[11pt]{article}
\usepackage[utf8]{inputenc}
\usepackage[T1]{fontenc}
\usepackage{graphicx}
\usepackage{grffile}
\usepackage{longtable}
\usepackage{wrapfig}
\usepackage{rotating}
\usepackage[normalem]{ulem}
\usepackage{amsmath}
\usepackage{textcomp}
\usepackage{amssymb}
\usepackage{capt-of}
\usepackage{hyperref}
\usepackage{minted}
\author{NTNU IDATA2302}
\date{Sep. 5, 2021}
\title{Efficiency, Algorithm Analysis, and Big-Oh Notation\\\medskip
\large Week 2 Quiz}
\hypersetup{
 pdfauthor={NTNU IDATA2302},
 pdftitle={Efficiency, Algorithm Analysis, and Big-Oh Notation},
 pdfkeywords={},
 pdfsubject={},
 pdfcreator={Emacs 27.2 (Org mode 9.4.4)}, 
 pdflang={English}}
\begin{document}

\maketitle


\section{Questions}
\label{sec:org648967b}

Here is a list of questions to assess our understanding of efficiency,
algorithm analysis and the Big-Oh notation. Keep in mind than several
answers can be correct.

\begin{enumerate}
\item How much time is required to execute the following RAM assembly
program?
\begin{minted}[]{asm}
load 0
add 15
sub 15
jump 0
\end{minted}

\begin{enumerate}
\item 4 units of time
\item 8 units of time
\item This program does not terminate
\item One cannot say
\end{enumerate}

\item Given the following Python program, how many times is the condition of the
\texttt{while} loop evaluated?
\begin{minted}[linenos,firstnumber=1]{python}
limit = 100
i = 50
while i < limit:
  print(i)
  i = i + 2
\end{minted}

\begin{enumerate}
\item 1 times
\item 25 times
\item 50 times
\item 100 times
\item Something else
\item This program does not terminate
\item One cannot say
\end{enumerate}

\item Provided that Algorithm A runs in \(O(n \log n)\) and Algorithm B runs in
\(O(n^2)\), is Algorithm A always faster than Algorithm B?
\begin{enumerate}
\item Yes
\item No
\item One cannot say
\end{enumerate}

\item Given a function \(f(x)\) such that \(f(x) = x^2 + 34x - 17\), is the
statement \(f \in \Theta(x^3)\) correct?
\begin{enumerate}
\item Yes
\item No
\item One cannot say
\end{enumerate}

\item Given the same function \(f(x)\), is the statement \(f(x) \in
   O(x!)\) correct?
\begin{enumerate}
\item Yes
\item No
\item One cannot say
\end{enumerate}

\item Given the same function \(f(x)\), is the statement \(f \in O(x \log x)\)
correct?
\begin{enumerate}
\item Yes
\item No
\item One cannot
\end{enumerate}

\item If I know that \(g \in O(x \log x)\), can I conclude that \(g \in
   \Theta(x^2)\)?
\begin{enumerate}
\item Yes
\item No
\item One cannot say
\end{enumerate}

\item Is it possible for a given algorithm to have its worst and best
cases in different efficiency class (e.g., \(n^2\) and \(n\))?
\begin{enumerate}
\item Yes
\item No
\item One cannot say
\end{enumerate}

\item If I read somewhere that the runtime of a given algorithm is \(\Theta(n)\),
what can I conclude?
\begin{enumerate}
\item It will \emph{always grow linearly} with respect to the problem size
\item It will \emph{never grow any faster} than linearly w.r.t to the problem size
\item It will \emph{never grow any slower} than linearly w.r.t to the problem size
\item One cannot say
\end{enumerate}

\item For any given algorithm, can the average case be \(O(n^2)\) whereas
the worse case is \(O(\sqrt{n})\)?
\begin{enumerate}
\item Yes
\item No
\item One cannot say
\end{enumerate}
\end{enumerate}



\section{Solutions}
\label{sec:org3486fee}

Here are the solutions I would choose, with brief explanations. Feel
free to reach out if anything is unclear or wrong.

\begin{enumerate}
\item \textbf{Answer (d): One cannot say.} We don't know where in memory this
program is stored. If \texttt{load 0} is stored at address 0, then the
program would not terminate. The \texttt{ACC} register will always be
zero so we will loop forever.

\item \textbf{Answer (e): Something else.} The loop will be evaluated 26 times,
basically for 50, 52, 53, etc. until the variable \texttt{i}
contains 102. Only then will the condition be false.

\item \textbf{Answer (c): One cannot say.} The big-Oh notation, dominance
relationships hold only beyond a specific problem size, which we
don't know in this case. However, it might very well be the case
that a function always dominate the other.

\item \textbf{Answer (b): No}. \(f \in \Theta(x^3)\) implies that \(f \in
     \Omega(x^3)\), which is incorrect. \(f \in O(x^3)\) is however
correct.

\item \textbf{Answer (a): Yes.} Since \(f \in O(x^3)\), \(f\) also admits any
other upper bound that grow faster than \(x^3\), including
\(O(x^4)\), \(O(2^x)\), or \(O(x!)\).

\item \textbf{Answer (b): No.} \(f \in \Theta(x^3)\) implies that \(f \in
     O(x^3)\). This is the tightest possible bound.

\item \textbf{Answer (b): No.} Big-O (the upper bound) is not sufficient to
establish Big-\(\Theta\) (in-the-order-of), because Big-\(\Theta\)
demands both the upper and the lower bound.

\item \textbf{Answer (a): Yes}. Nothing prevents this.

\item \textbf{Answer (a), (b), and (c)}. If will always grow linearly, which
implies that it never grows any faster and any slower than
linearly.

\item \textbf{Answer (b): No}. The average case cannot be worse than the
worst case.
\end{enumerate}
\end{document}