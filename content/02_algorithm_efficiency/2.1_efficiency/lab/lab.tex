% Created 2021-09-11 Sat 06:02
% Intended LaTeX compiler: pdflatex
\documentclass[11pt]{article}
\usepackage[utf8]{inputenc}
\usepackage[T1]{fontenc}
\usepackage{graphicx}
\usepackage{grffile}
\usepackage{longtable}
\usepackage{wrapfig}
\usepackage{rotating}
\usepackage[normalem]{ulem}
\usepackage{amsmath}
\usepackage{textcomp}
\usepackage{amssymb}
\usepackage{capt-of}
\usepackage{hyperref}
\usepackage{minted}
\setcounter{secnumdepth}{2}
\author{NTNU IDATA 2302}
\date{Aug. 30, 2021}
\title{Algorithm Efficiency\\\medskip
\large Lab Session 2.1}
\hypersetup{
 pdfauthor={NTNU IDATA 2302},
 pdftitle={Algorithm Efficiency},
 pdfkeywords={},
 pdfsubject={},
 pdfcreator={Emacs 27.2 (Org mode 9.4.4)}, 
 pdflang={English}}
\begin{document}

\maketitle
\tableofcontents


\section{Introduction}
\label{sec:org268a817}

In Week 1, we defined what is an algorithm (as opposed to a program)
and we looked at techniques to assess correctness. We now turn to
efficiency, which is one of the core topic of this course. For
the record, efficiency relates to two resources: Runtime and
memory. Runtime captures the time the underlying RAM architecture needs to
run an algorithm until it terminates. Memory reflects how many memory
cells the algorithm needs beyond what it uses to store the
algorithm's instructions.

In this lab session, we will practice evaluating runtime and memory
consumption. We will start with RAM assembly programs, and then move
on to higher-level languages such as Java, C, Python and the likes. We
will look at three short algorithms that illustrate different control
structures:

\begin{enumerate}
\item \emph{Adding Three Numbers} is a short program that reads three
numbers from the user and print their sum. This illustrates the
basic concepts.
\item \emph{Maximum} is a program that accepts two numbers and prints the
largest one. It shows how to evaluate the runtime and memory of a
conditional statement.
\item \emph{Multiplication} is a program that prints the product of two numbers,
but unfolds the multiplication as a series of additions using a
loop. It shows how to evaluate the runtime and memory of loops.
\end{enumerate}

\section{Adding Three Numbers}
\label{sec:orgd3a0041}

\subsection{From RAM Assembly}
\label{sec:orgfa42d15}

Let us start with a short RAM assembly program, which reads three
number from the user and outputs their sum. Listing
\ref{org4a6020b} shows the corresponding RAM assembly code. 

\begin{listing}[htbp]
\begin{minted}[linenos,firstnumber=1]{asm}
SEGMENT: data       ; The variables
      value 1 0       ; value is 1 memory cell initialized at 0
      total 1 0       ; total is 1 memory cell initialized at 0  

SEGMENT: code       ; The instructions
      load  0         ; ACC := 0
      read  value     ; value := user-input
      add   value     ; ACC := ACC + value
      read  value     ; value := user-input
      add   value     ; ACC := ACC + value
      read  value     ; value := user-input
      add   value     ; ACC := ACC + value
      store total     ; total := ACC
      print total     ; 
      halt  -1        ;
\end{minted}
\caption{\label{org4a6020b}A RAM assembly program that adds up three numbers given by the user.}
\end{listing}

\textbf{The whole point is to evaluate---without running it---the time and
memory it would take on a RAM}. Note that this is only \emph{one possible
program}: Others are possible. One could for instance reuse the
\texttt{value} variable to store the result (instead of another \texttt{total}
variable). In practice, such choices are made by the compiler or
interpreter and we will later work directly from higher-level
languages.

\subsubsection*{Memory}
\label{sec:orgc365fd8}

In Lecture 2.1, we defined \emph{memory consumption} as the number of
memory cells that we need to run our algorithm. How many do we need
here? First, let us look at the \texttt{code} segment, which contains 10
instructions, and, since each instruction takes 2 memory cells, that
is a total of 20 cells. In practice, we will not account for the code
segment, so we can just omit this. What remains is the data segment,
where we declared two variables, namely \texttt{value} and \texttt{total}. The
\texttt{value} variable is however used to hold user inputs, and such
variables are often excluded, so that gives us \textbf{a total of 1 memory
cells}.

\subsubsection*{Runtime}
\label{sec:orgd62d916}

Moving now to the runtime. It captures how much time our program will
take on a RAM architecture. To estimate it, we must know two things:
\begin{itemize}
\item How much time each RAM instruction takes to run;
\item and how many times it will run.
\end{itemize}

We therefore need a \emph{cost model}, which is our assumptions about the
time each RAM instruction takes to complete. For the sake of
simplicity, I will assume (as in most textbooks) that every
instruction takes 1 single unit of time.

Now, if we look at Listing \ref{org4a6020b}, we see that it
does not contain any \texttt{jump} instruction: The machine start with \texttt{LOAD
0} (Line 6) and ends with \texttt{HALT -1} (Line 15). Each of the 10
instructions will run once, so \textbf{it takes exactly 10 units of time}.

\subsection{From Java}
\label{sec:org2e40006}

Let us now look at higher-level languages. How can we evaluate the
runtime and memory of Java programs without compiling them explicitly
to RAM assembly?

Listing \ref{org62d7339} shows a Java program, equivalent to the
RAM assembly program that adds three numbers given by the user (see
Listing \ref{org4a6020b}).

\begin{listing}[htbp]
\begin{minted}[linenos,firstnumber=1]{java}
public int main(String[] args) {
    Scanner input = new Scanner(System.in);
    int total = 0;
    total = total + input.nextInt();
    total = total + input.nextInt();
    total = total + input.nextInt();
    System.out.println(total);
}
\end{minted}
\caption{\label{org62d7339}A Java program that adds three numbers given by the user}
\end{listing}

\subsubsection*{Memory}
\label{sec:orge505d1e}

To estimate the memory consumption of a Java program, we simply count
the number of variables it uses. In this case, there are two,
namely \texttt{input} and \texttt{total}. The \texttt{input} variable (see Line 2) is
not part of our algorithm: It is only some "plumbing" needed in the
Java implementation, so we are left \textbf{with one single variable}.

\subsubsection*{Runtime}
\label{sec:org68bc6af}

Now, if we look at the runtime, we need a way to estimate the time a
Java instruction would take on a RAM architecture. Again, we need the
same two things:
\begin{enumerate}
\item How much time is spent executing each Java statement, the cost per
statement;
\item The number of times each line is executed, its frequency count.
\end{enumerate}

As for the cost per statement, we look at every statement and estimate
the minimum number of RAM instructions needed. Note that the statement
\texttt{int total=0} is a variable declaration and translate into the data
segment, so we skip it. The Java statement "\texttt{total = total +
input.nextInt()}" translates into at least three instructions: a
\texttt{READ}, an \texttt{ADD}, and a \texttt{STORE}. By contrast, the Java statement
"\texttt{System.out.println(total)}" (see Line 7) translates into at
least a \texttt{PRINT} instruction. Table \ref{tab:orga7b6e18} summarizes the
cost of each line.

Regarding the frequency count, Listing \ref{org62d7339} does not
include any control structures (i.e., loops, conditionals, or goto
statement), so each statement is executed only once. Table
\ref{tab:orga7b6e18} summarizes the frequencies of each statement.

\begin{table}[htbp]
\caption{\label{tab:orga7b6e18}Summary of lines' costs and frequency of Listing \ref{org62d7339}}
\centering
\begin{tabular}{lrrrr}
Java Code & Line & Cost &  Freq. & Total\\
\hline
\texttt{total = total + input.nextInt()} & 4 & 3 & 1 & 3\\
\texttt{total = total + input.nextInt()} & 5 & 3 & 1 & 3\\
\texttt{total = total + input.nextInt()} & 6 & 3 & 1 & 3\\
\texttt{System.out.println(total)} & 7 & 1 & 1 & 1\\
\hline
 &  &  & \textbf{Grand Total} & 10\\
\end{tabular}
\end{table}

Finally we can compute the total runtime of our algorithm by summing
up all these costs as follows:

\begin{align*}
  \text{total time} &= (3 \times 1) + (3 \times 1) + (3 \times 1) + (1 \times 1) \\
                    &= 10 \\
\end{align*}

Here we get the same evaluation as for the RAM assembly program, but
it is only because this program does not contain any control structure.

\section{Maximum}
\label{sec:org4730fc3}

Now, we will look at a conditional statement (i.e., \texttt{if-then-else}
in Java). As before, we will start with a RAM assembly program and
move on to the same Java program.

\subsection{From RAM Assembly}
\label{sec:org63d4215}

Listing \ref{org9420b65} shows one possible RAM assembly program that
selects the maximum between two values. Note that these values, named
\texttt{left} and \texttt{right} are not given by the user, but instead predefined
in the data segment (see Lines 2 and 3). We will see in the
next lecture how to account for unknown values.

To compare these \texttt{left} and \texttt{right} variables, we check whether their
difference is negative (see Line 10)\footnote{Keep in mind that the original RAM architecture only offers one
single \texttt{JUMP} instruction, which jumps only when the \texttt{ACC} register is
positive or null.}. In that case, we
jump to the \texttt{lmax} label where we assign \texttt{maximum} with \texttt{left} and
print it. Otherwise. we assign \texttt{maximum} with right and we jump to the
\texttt{done} label (see Line 19) where we print the \texttt{maximum}
variable.

\begin{listing}[htbp]
\begin{minted}[linenos,firstnumber=1]{asm}
SEGMENT: data           ; Variables
        left    1 25      ; left := 25
        right   1 45      ; right := 45
        maximum 1 0       ; maximum := 0

SEGMENT: code          ; Instructions
        load 0         ;
        add left       ;
        subtract right ;
        jump lmax      ; if left - right < 0
        load 0         ;
        add right      ;
        store maximum  ;   maximum := right 
        load 0         ;
        jump done      ;
lmax:   load 0         ; else
        add left       ;
        store maximum  ;   maximum := left
done:   print maximum  ; print(maximum)
        halt -1
\end{minted}
\caption{\label{org9420b65}RAM program that prints the largest of two predefined numbers}
\end{listing}

\subsubsection*{Memory}
\label{sec:orgb3267a2}

Evaluating the memory consumption boils down to counting the variables
our program uses. There are three, \texttt{left}, \texttt{right} and \texttt{maximum}. It
is customary to omit the user inputs, that is \texttt{left} and \texttt{right}
here. We are left with 1 variable.

\subsubsection*{Runtime}
\label{sec:orgc3f9e47}

As for the runtime of such a conditional statement, only one part is
executed, depending on how the condition evaluates. In this case, we
do know the value of the \texttt{left} and \texttt{right} variables (25 and 45,
respectively), so we know which branch will be executed. At line
10, the \texttt{ACC} register contains \(25 - 45 = -20\), and the
machine does \emph{not} jump to \texttt{lmax}, but proceeds with the next
instruction. Once it reaches Line 15, it jumps directly
to the label \texttt{done} (because \texttt{ACC} was explicitly set to 0
before). The block starting at label \texttt{lmax} is thus never executed.

In total, the machine only executes 11 instructions: It starts at
\texttt{LOAD 0} (Line 7) and continues until \texttt{JUMP done} (9
instructions until Line 15). It then directly jumps to
the label \texttt{done} (Line 19 and continues from there to the end
(2 instructions).

Provided a cost model, where every RAM instruction takes 1 unit of time,
these 11 instructions take 11 units of time to run.

\subsection{From Java}
\label{sec:orgcb046f9}

Again, let us look at the same algorithm but implemented in Java this
time. Listing \ref{org0d13a30} shows the Java code, which boils down
to a condition statement followed by a print statement.

\begin{listing}[htbp]
\begin{minted}[linenos,firstnumber=1]{java}
public static void main(String[] args) {
    int left = 25;
    int right = 45

    int maximum = 0;
    if (left < right) {
        maximum = right;
    } else {
        maximum = left;
    }

    System.out.println(maximum);
}
\end{minted}
\caption{\label{org0d13a30}A Java implementation of the maximum algorithm (cf. Listing \ref{org9420b65})}
\end{listing}

\subsubsection*{Memory}
\label{sec:org7c03d41}

The memory is very similar to the RAM assembly program. The Java code
from Listing \ref{org0d13a30} defines three variables, two of which
are inputs (\texttt{left} and \texttt{right}). So we are left with one variable.

\subsubsection*{Runtime}
\label{sec:orgde55c23}

Regarding the runtime, we proceed as we did in for adding three
numbers (see Section \ref{sec:orgd3a0041}). We first estimate the cost
of each Java statement by translating it into a set of RAM
instructions. Then, we calculate how many times each statement runs.

Listing \ref{org0d13a30} contains four different types of statements,
namely \emph{variable declaration}, \emph{assignment}, \emph{conditional} and
\emph{output}. Let us look at them in turn:

\begin{itemize}
\item Variable declarations translate into the data segment, so there is
no need to account for them.

\item Evaluating a condition (e.g., Line 6) translates into
at least an arithmetic operation (e.g., \texttt{SUBTRACT}) and one \texttt{JUMP}.
So the cost is 2

\item Assignments translate into a single \texttt{STORE} instruction, so they
cost 1.

\item Outputs translate into a single \texttt{PRINT} statement, so they also cost 1
unit of time.
\end{itemize}

Turning to the frequencies, the \texttt{else} branch will never be executed,
because \texttt{left} is less than \texttt{right}. All other instructions will be
executed only once. Table \ref{tab:orgd479692} summarizes these costs and
frequencies. In total, we are left with the following:

\begin{align*}
   \text{time} &= (2 \times 1) + (1 \times 1) + (1 \times 1) \\
               &= 4 \\
\end{align*}

Here we get a results that further away from what we calculated for
the equivalent RAM program. The difference comes from the conditional
statement, which in practice include more instructions than the two
we accounted for, but this depends on the compiler in practice.

\begin{table}[htbp]
\caption{\label{tab:orgd479692}Summary of the cost and frequencies of Listing \ref{org0d13a30}}
\centering
\begin{tabular}{rlrrr}
Line & Java Code & Cost & Freq.  &  Total\\
\hline
6 & \texttt{if (left < right) \{} & 2 & 1 & 2\\
7 & \texttt{maximum = right;} & 1 & 1 & 1\\
8 & \texttt{\} else \{} & 0 & 0 & 0\\
9 & \texttt{maximum = left;} & 1 & 0 & 0\\
10 & \texttt{\}} & 0 & 1 & 0\\
12 & \texttt{System.out.println(maximum)} & 1 & 1 & 1\\
\hline
 &  &  & \textbf{Total} & 4\\
\end{tabular}
\end{table}


\section{Multiplication}
\label{sec:org09cba21}

Finally, we now look at a program that contains a loop. This program
unfolds a multiplication as a series of additions. \emph{This is useless in
practice, and only illustrates the concept of loop}.

\subsection{From RAM assembly}
\label{sec:orgb4863df}

Listing \ref{org6ba9972} shows a RAM assembly program that
implements such a multiplication. We first initialize a \texttt{counter}
variable to zero (see the data segment). When the program starts, it
first computes the difference between the \texttt{counter} and the
\texttt{multiplier} variables. If this is greater or equals to zero, the
program jumps at Line 21 and prints the \texttt{product}. Otherwise
the program increments the \texttt{product} with the \texttt{multiplicand},
increment the \texttt{counter} by 1, and continues at Line 8.

\begin{listing}[htbp]
\begin{minted}[linenos,firstnumber=1]{asm}
segment: data
        multiplicand  1   10       ; the multiplicand = 10
        multiplier    1    4       ; the multiplier = 4
        counter       1    0       ; the loop index
        product       1    0       ; the product

segment: code
loop:   load          0            ;
        add           counter      ;
        subtract      multiplier   ; ACC := counter - multiplier 
        jump          done         ; if ACC >= 0, goto done
        load          0            ;
        add           product      ;
        add           multiplicand ;
        store         product      ; product += multiplicand
        load          1            ;
        add           counter      ;
        store         counter      ; counter += 1
        load          0            ;
        jump          loop         ; goto loop
done:   print         product      ;
        halt          -1
\end{minted}
\caption{\label{org6ba9972}A RAM assembly program that multiplies two numbers predefined in memory}
\end{listing}

\subsubsection*{Memory}
\label{sec:org5489f67}

Listing \ref{org6ba9972} declares four variables (see the data
segment), each taking one memory cell. If we omit the two inputs (i.e.,
\texttt{multiplier} and \texttt{multiplicand}), we are left with two memory cells.

\subsubsection*{Runtime}
\label{sec:org84bfcba}

To evaluate the runtime of Listing \ref{org6ba9972}, we will
assume a cost where every RAM instruction takes one unit of time. The
challenge with loops is to understand how many time the body runs.

In Listing \ref{org6ba9972}, Lines 8 to 11
test whether we should exit the loop. These 4 lines will be executed
once every time the condition holds, but also once when \texttt{counter}
exceeds \texttt{multiplier}. Since we know that \texttt{multipler} is 4, these lines
will execute 5 times. The body of loop (i.e., the 9 instructions from
Lines 12 to 20) will run each time the test
condition holds, that is, 4 times. Finally, the last two instructions
(from Line 21) will be executed once. We are thus left with
the following:

\begin{align*}
 \text{time} &= (5 \times 4) + (4 \times 9) + 2 \\
             &= 20 + 36 + 2 \\
             &= 58
\end{align*}

Listing \ref{org6ba9972} takes 58 units of time to execute.

\subsection{From Java}
\label{sec:org700d220}

Finally, we can now look at an equivalent Java program. In Listing
\ref{orge874f22}, we first defined the \texttt{multiplicand} and the
\texttt{multiplier} as two integer variables, and we then use a \texttt{while} loop
to add the \texttt{multiplicand} many times.

\begin{listing}[htbp]
\begin{minted}[linenos,firstnumber=1]{java}
public static void main(String[] args) {
    final int multiplicand = 10;
    final int multiplier = 4;

    int product = 0;
    int counter = 0;
    while (counter < multiplier) {
        product = product + multiplicand;
        counter = counter + 1;
    }

    System.out.println(product);
}
\end{minted}
\caption{\label{orge874f22}A Java program that multiplies two predefined variables (cf. Listing \ref{org6ba9972})}
\end{listing}

\subsubsection*{Memory}
\label{sec:org96cd4c8}

Listing \ref{orge874f22} declares four variables, namely
\texttt{multiplicand}, \texttt{multiplier}, \texttt{product}, and \texttt{counter}. Two of them
are inputs, so we are left with two as a memory consumption.

\subsubsection*{Runtime}
\label{sec:org3a6c6a6}

As we did before, we need to first estimate the cost of each Java
statement and then calculate how many times each statement is
executed.

If we put aside variable declarations, we are left with a loop whose
body contains two assignments.

\begin{itemize}
\item Assignments translate into RAM assembly as one \texttt{STORE}
instructions. Here however, both assignments include an arithmetic
expression, which we should account for, because it translates into
an \texttt{ADD} instruction. Such assignments thus cost 2 units of time.

\item The loop and its condition take one comparison and one \texttt{jump}
instructions, for a cost of 2. Note that the loop body includes an
extra jump to return at the beginning of the loop.

\item The print statements translates to a single \texttt{PRINT} instruction.
\end{itemize}

Regarding the frequencies, we know that the loop condition will be
executed as many times as the condition holds, and one time when the
condition breaks. Here that is 5 times. By contrast, the loop body
executes only when condition holds, that is, 4 times in our case. Table
\ref{tab:orgb0cd28a} summarizes these costs and frequencies.

Eventually we can model the time required by Listing
\ref{org6ba9972} as follows:

\begin{align*}
 \text{time} & = (5 \times 2) + [ 4 \times (2 + 2 + 1) ] + 1 \\
             & = 10 + 20 + 1 \\
             & = 31 \\
\end{align*}

Here, again, the estimation we get differ from the one we got for the
RAM assembly. The difference is wider because our errors are amplified
each time the loop runs.

\begin{table}[htbp]
\caption{\label{tab:orgb0cd28a}Summary of the cost and frequencies of Listing \ref{orge874f22}}
\centering
\begin{tabular}{rlrrr}
Line & Java Code & Cost & Freq.  &  Total\\
\hline
6 & \texttt{while (counter < multiplier) \{} & 2 & 5 & 10\\
7 & \texttt{product = product + multiplicand;} & 2 & 4 & 8\\
9 & \texttt{counter = counter + 1;} & 2 & 4 & 8\\
10 & \texttt{\}} & 1 & 4 & 4\\
12 & \texttt{System.out.println(product)} & 1 & 1 & 1\\
\hline
 &  &  & \textbf{Total} & 31\\
\end{tabular}
\end{table}


\section{Conclusions}
\label{sec:org09123de}

The first take-away from this lab session is that, in practice,
estimating the cost of a Java statement boils down to counting
arithmetic operations, comparisons, logical operations, and
assignment. As for the memory consumption, we simply count the
variables.

The second take-away is that we only \textbf{estimate} the runtime and memory
consumption. Since we do not know exactly how our code will be
compiled to a RAM architecture, we cannot know the exact number of
memory cells nor the number of instructions executed. We can however
get an estimation, which, as we shall see in the next lectures, is
good enough to compare algorithms.
\end{document}