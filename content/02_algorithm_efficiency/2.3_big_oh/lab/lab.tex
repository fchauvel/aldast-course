% Created 2021-09-04 Sat 07:43
% Intended LaTeX compiler: pdflatex
\documentclass[11pt]{article}
\usepackage[utf8]{inputenc}
\usepackage[T1]{fontenc}
\usepackage{graphicx}
\usepackage{grffile}
\usepackage{longtable}
\usepackage{wrapfig}
\usepackage{rotating}
\usepackage[normalem]{ulem}
\usepackage{amsmath}
\usepackage{textcomp}
\usepackage{amssymb}
\usepackage{capt-of}
\usepackage{hyperref}
\usepackage[cache=false]{minted}
\author{NTNU IDATA 2302}
\date{Sep. 2, 2021}
\title{Big-Oh Notation\\\medskip
\large Lab Session 2.3}
\hypersetup{
 pdfauthor={NTNU IDATA 2302},
 pdftitle={Big-Oh Notation},
 pdfkeywords={},
 pdfsubject={},
 pdfcreator={Emacs 27.2 (Org mode 9.4.4)}, 
 pdflang={English}}
\begin{document}

\maketitle
\tableofcontents


\section{Introduction}
\label{sec:org06c48dc}

Now we have seen all the needed foundations. First we looked at
programs and algorithms, and how the \emph{RAM architecture} helps us
understand what is happening. We also precisely defined \emph{runtime} and
\emph{memory consumption} and we learned about \emph{algorithm analysis} to
define the best, worst, and average cases. In today's lecture, we
studied the \emph{big-Oh notation}, which we use to obtain a very abstract
description of the relationship between the problem size and the
efficiency of our algorithms.

Now, we are fully equipped to go all the way through a simple
example. There will be many more example throughout the rest of the
course.

For this lab session, I suggest we look at a Java program that draws a
triangle in ASCII. Listing \ref{orge69745b} shows one possible Java
implementation. It uses two \emph{nested} loops. The outer loop iterates
through the rows to be printed (from line 4), whereas the nested
loop iterates through a single line (from Line 5). One way to
think of it, is that we print one "\texttt{*}" on the first line, two "\texttt{*}"
on the second lines, three "\texttt{*}" on the third line, and so on and so
forth.

\begin{listing}[htbp]
\begin{minted}[linenos,firstnumber=1]{java}
public class Triangle {

    public static void printTriangle(int height) {
        for(int row=0 ; row<height ; row++){
            for(int column=0 ; column<=row ; column++){
                System.out.print("* ");  
            }  

            System.out.println();
        } 
    }

    public static void main(String[] args) {  
        printTriangle(6);
    }  
}  
\end{minted}
\caption{\label{orge69745b}Drawing triangles on the console in Java}
\end{listing}

Should you compile and run this program, you would see something like:
\begin{verbatim}
* 
* * 
* * * 
* * * * 
* * * * * 
* * * * * * 
\end{verbatim}

\section{Algorithm Analysis}
\label{sec:org176cceb}

\subsection{What is the \emph{Problem Size}?}
\label{sec:orgb5cd914}

The first step is to find the size of the problem. The size of our
problem surfaces in the output of our program, as the number of "\texttt{*}"
we print. So what governs this number of stars? It is the
height of the triangle: The higher the triangle, the more stars we
need to print. This is the size of our problem\footnote{Theoretically, the size of the problem is defined as the number
of bits needed to encoded the inputs.}.

\subsection{Worst, Best, and Average cases}
\label{sec:orgd0ddf91}

Remember the definition of worst, best and average cases. They
characterize, for a given problem size, different situations that
our algorithm may face. Here however, there is nothing besides the
given height that one needs to know to print the triangle. In this
case, the best, the worst and the average case are the same.

\subsection{Modeling Runtime efficiency}
\label{sec:org59f890e}

Since there is no difference between the best, worse and average,
our job is simpler and a single efficiency model suffices. Now, we
need to find a function that captures how the runtime responds to
changes in the problem size.

I found our program, with its two nested \texttt{for}-loops, difficult to
work with (see Listing \ref{orge69745b}). When I tried to directly
find the time spent on each line and the number of time each line is
executed, I struggled to express the costs and frequencies of lines
4 to 9. So to make our life easier, I rewrite it so
that:

\begin{enumerate}
\item All \texttt{for} loops are expanded into \texttt{while} loops;
\item Nested loops are expanded into separate functions, which we can
analyze separately.
\end{enumerate}

Listing \ref{orge0b5748} shows the results of this
rewriting. We can now proceed and by finding the cost (time) and the
frequency of each line.

\begin{listing}[htbp]
\begin{minted}[]{java}
public static void printTriangle(int height)
{                              
    int row = 0;               
    while (row < height) {     
        printRow(row); // New function call
        row = row + 1;         
    }
}

// Contains the code of the original inner for-loop
public static void printRow(int row)
{                                  
    int column = 0;                
    while (column <= row) {        
        System.out.print("* ");    
        column = column + 1;       
    }                              
    System.out.println();          
}
\end{minted}
\caption{\label{orge0b5748}Expanding \texttt{for} loops and extracting the inner loop as a separate function}
\end{listing}

Let us start with the inner loop, that is, the new function
\texttt{printRow}. We look at each line, and we figure out how much time it
would take to run it on a RAM architecture (its cost), and how many
time it will be executed (its frequency). The total cost of a line
would hence be given by:

\[
     \text{total cost} = \text{cost} \times \text{frequency}
  \]

Looking at the \texttt{printRow} function line by line we get:

\begin{enumerate}
\item The first assignment would translate into at least a \texttt{store}
instruction, and will be executed once. That gives a total cost
of \(1 = 1 \times 1\).

\item The condition \texttt{column <= row} contains a single comparison, which
would translate to at least a \texttt{jump} instruction in RAM
assembly. This condition will be evaluated as many times as it
holds, and one more time when it will not hold anymore. That
gives us a total cost of \(1 \cdot (r + 2)\) where \(r\) denotes the
\texttt{row} parameter.

\item The print statement maps to a \texttt{print} instruction in RAM assembly
and it will be executed as many times as we enter the loop, that
is, \(r+1\) times. So we get a total cost of \(1 \cdot (r+1)\).

\item Incrementing the \texttt{column} variable by 1 takes at least two RAM
instructions: an \texttt{add} and a \texttt{store}. It will be executed as many
time as we enter the loop, for a total cost of \(2 \cdot (r +
     1)\).

\item Finally printing a new line translates to a single \texttt{print}
instruction and is executed once. That gives us a total cost of
\(1 \times 1\).
\end{enumerate}

All together, the time it takes for our \texttt{printRow} function to run
is given by the sum of the total cost of each line. We can simplify
this as follows:

\begin{align*}
time_R(r) &= [1 \times 1] + [1 \cdot (r + 2)] + [1 \cdot (r+1)] + [2 \cdot (r + 1)] + [1 \times 1] \\
          &= 1 + (r + 2) + (r + 1) + (2r + 2) + 1 \\
          &= 4r + 7 
\end{align*}

Now we can now look to the \texttt{printTriangle} function (i.e., the outer
loop of our original program), proceeding as we did previously. We
get:

\begin{enumerate}
\item The first assignment translates into a single \texttt{store} instruction
on the RAM architecture, and is executed once. So we get a total
cost of \(1 \times 1\).

\item The condition \texttt{row < height} translates in at least a \texttt{jump}
instruction and is executed as many times as it holds and one more
time when it will not hold. Since the \texttt{row} variable starts at
zero and ends \emph{before} \texttt{row} equals \texttt{height}, it will run \(h + 1\)
where \(h\) denote the \texttt{height} parameter. So we get a total cost
of \(1 \cdot (h+1)\).

\item What about this function call we have introduced? We know it will
runs as many times as we enter the loop, but how much does it
cost?  The cost is given by function \(time_R(r)\), which we sorted
out earlier. The challenge is that the \texttt{row} variable changes
value each time the loop body runs. So the total cost is given
by:

\begin{align*}
    cost(call) & = time_R(0) + time_R(1) + \ldots + time_R(h-1) \\
        & = \sum_{r=0}^{h-1} time_R(r) \\
        & = \sum_{r=0}^{h-1} 4r + 7 \\
        & = \sum_{r=0}^{h-1} 4r + \sum_{r=0}^{h-1} 7 \\
        & = 4 \cdot \sum_{r=0}^{h-1} r + \sum_{r=0}^{h-1} 7 \\
        & = 4 \cdot \sum_{r=0}^{h-1} r + 7h \\
        & = 4 \cdot \frac{(h-1)h}{2} + 7h \\
        & = 2 \cdot [h^2 - h] + 7h \\
        & = 2h^2 - 2h + 7h \\
        & = 2h^2 + 5h \\
\end{align*}

\item Incrementing \texttt{row} translates into two RAM instructions (\texttt{store}
and \texttt{add}), and it runs every time we enter the loop. We get a total
cost of \(2 \times h\)
\end{enumerate}

We can add all these costs together to get the total runtime of our
\texttt{printTriangle} function, as follows:

\begin{align*}
time_T(h) &= [1 \cdot 1] + [1 \cdot (h + 1)] + [2h^2 + 5h] + [2 \cdot h] \\
          &= 1 + (h+1) + (2h^2 + 5h) + 2h \\
          &= 2h^2 + 8h + 2
\end{align*}

\section{Classification}
\label{sec:orgcd20c37}

The last step consists in classifying the efficiency of our
algorithm using the Big-Oh notation. To do that we can try to find its
upper and lower bounds.

\subsection{Finding an upper bound}
\label{sec:org89ad828}

Let us start with the upper bound. Recall the definition of an upper
bound using the Big-Oh notation:

\[
   time_T \in O(g) \iff \\
   \exists \, (c, n) \in \mathbb{N}^2, \quad \forall n \geq n_0, \; time_T(n) \leq c \cdot g(n)
   \]

What could be this function \(g(n)\) and these constant \(c\) and
\(n_0\)? If we start from our efficiency model \(time_T(h) = 2h^2 +
   8h + 2\) and we drop the less important terms, we get \(g(n) =
   n^2\). Let us try to see if we can find a constant c and \(n_0\) that
matches our definition. Let us try \(c=3\), the first integer value
greater than 2:

\begin{align*}
time_T(n) & \leq c \cdot n^2 \\
2n^2 + 8n + 2 & \leq 3n^2 \\
2 & \leq n^2 - 8n \\
2 & \leq n (n - 8) \\
9 & \leq n \\
\end{align*}

That gives us our value for \(n_0\). So we can conclude that our
runtime efficiency model admits a quadratic upper bound, that is
\(time_T(n) \in O(n^2)\).

\subsection{Finding a lower bound}
\label{sec:org8ede382}

We can now look at the lower bound. Recall that the definition of a
lower bound is:

\[
   time_T \in \Omega(g) \iff \\
   \exists \, (c, n) \in \mathbb{N}^2, \quad \forall n \geq n_0, \; time_T(n) \geq c \cdot g(n)
   \]

We can proceed as we did for the upper bound. We can simply solve a
similar inequality where we can assume that \(c=2\) as follows:

\begin{align*}
time_T(n) & \leq c \cdot n^2 \\
2n^2 + 8n + 2 & \leq 2n^2 \\
2 & \leq  8n \\
\frac{2}{8} & \leq n \\
0.25 & \leq n \\
\end{align*}

That gives us again the value for \(n_0\). We can conclude that our
runtime efficiency model admits a quadratic lower bound, that
\(time_T(n) \in \Omega(n^2)\).

Finally, since our upper bound holds from \(n=9\) and our lower bound
from \(n = 0.25\), we can conclude that our lower bound also holds
from \(n=9\) and beyond. Our efficiency model is therefore "in the
order of quadratic function", or in other words, \(time_T(n) \in
   \Theta(n^2)\).

\subsection{Visualization}
\label{sec:orgb97d692}

I like to look at efficiency models and bounds visually, as a
"sanity check".  Figure \ref{fig:orgf14afb6} shows our efficiency model, its
upper bound \(3 \cdot n^2\) and its lower bound \(2n^2\). Keep in mind
these are not the only possible ones.

The conclusion is that our \texttt{printTriangle} algorithm is always in
the order of a quadratic runtime: Worst, best and average case are
the same.

\begin{figure}[htbp]
\centering
\includegraphics[width=.9\linewidth]{./visualization.pdf}
\caption{\label{fig:orgf14afb6}Visualization of our runtime efficiency model, together with its lower and upper bound.}
\end{figure}
\end{document}