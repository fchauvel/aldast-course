\documentclass{aldast}

\usepackage{skull}

\documentType{Lecture Notes}
\documentNumber{1.1}
\title{Getting Started}
\author{F. Chauvel}


\begin{document}

\maketitle

\begin{abstract}
  Let us first clarify a few concepts: Computation, data, algorithm,
  and data structure. This will allow us to discuss the course
  syllabus, including the topics we will we cover and those we will
  not. We also look at expectations: What do I expect from you and
  what you can expect from me. I then go through the course
  organizations, with lectures, lab sessions, weekly quizzes and
  examinations. Finally, we look at external resources (books, online
  courses, websites, etc.) which might help.
\end{abstract}


\section*{Introduction}
Welcome! I\sidenote{My name is Franck. I am a full stack developer at
  Axbit AS and a lecturer at NTNU. Feel free to
  \href{mailto:franck.chauvel@ntnu.no}{reach out}.} am very happy to
welcome you in this course about \emph{Algorithms \& Data
  Structures}. I prepared this course as well as I could with the
time, resources and energy I had, and I do hope you will find it
relevant. Many thanks to the students who took this course in the
past: Their feedback has been very valuable. Let me know what you
think!

Despite its official title, this course is first and foremost about
\emph{solving ``programming'' problems}. This is the most valuable
tool a programmer has because it applies across technologies,
languages and frameworks. The things I learned 15 years ago still apply
to my daily engineering work. This is \emph{rock-solid}. But it is
also a competence that is \emph{hard to get}.

I would like to kick-off this course by presenting three core
concepts: Computation, algorithms, and data structures. From there, I
want to go through a few some practicalities, including learning
outcomes, expectations and supporting material, and examinations.


\section{Computation}
Computer Science (CS) is all about computers---as the name says. The
term ``computer'' stands however for anything that ``computes'', be it
a person, a machine, or something else. What CS is really about is
\emph{computation}. But what is that? A computation is a
transformation of \emph{data}: It consumes some data and produces some
data. That leads us directly to next question: What is \emph{data}?

Data is an overloaded term. Intuitively, it stands for more or less
anything: Texts, numbers, pictures, audio recording, etc. In CS
however, data is anything that we can write down using \emph{a finite
  set of symbols}. For example, the word ``computation'' is a sequence
of letters, each drawn from the Latin alphabet. Numbers shows up as
combinations of Arabic digits. Your smart phone computes using
``binary'' data, encoded using binary alphabet (0 or 1) \dots These are
various forms of data.

\paragraph{Symbols vs. Concepts} Symbols are things that stands for
something else, often a more ``abtsract'' thing. For example, in the
Western civilization, the symbol $\skull$ stands for danger or
death. Numbers are another example of ``concepts'' that we can write
using a variety of symbols. We can write the concept of ``two tens''
as a combination of Arabic digits ``20'', as the combination of Roman
digits ``XX'', or even with ``14'' in hexadecimal. We can also use
Latin alphabet like in ``twenty'', etc. All those represent the same
number. CS is only concerned with symbols and not with their
interpretation, which is peculiar to human cognition.

\begin{takeaway}
  A \emph{computation} transforms of a sequence of symbols into
  another.
\end{takeaway}

\subsection{Computational Problems}
We use computations to solve problems, but only some problems can be
solved by a computation! These are so-called \emph{computational
  problems}, and are the focus of this course.

A computational problem defines what are the valid outputs for any
given input given. Such a problem sets the requirements
for a computation. Here is an example:

\begin{align}
  add: \mathbb{N} \times \mathbb{N} &\mapsto \mathbb{N} \nonumber \\
  add\, (x, y) & \mapsto x + y
  \label{eq:addition}
\end{align}

The problem specified by Equation~\ref{eq:addition} is to find the
number that is the sum of the two given inputs $x$ and $y$. Note that
the '$+$' symbol I used above does not tell us how to actually carry
out an addition. It simply denotes the concept of ``addition'' and I
assume that everyone recognizes it and know what it means
(associativity, neutral element, etc.). I bet you already know a
procedure to find the sum of two natural numbers. In general, problems
describe relations between concepts: natural numbers, addition,
etc. They say nothing about the symbols to use.

There are many types of computational problems. In \emph{decision
  problems} for example, the output is a Boolean value (yes or no),
such as testing whether a given number is a prime number. We will also
meet search problems, where we have to find a particular value in a
larger set, such as finding all the pairs of numbers whose sum is
10. Another type of problem is counting, where we try to figure out
the number of solutions to a another problem. Finally there is also
optimization problems, where we search for the solution that maximizes
a specific criterion. All these problems can be described as
mathematical functions\sidenote{The concept of ``function'' exists
  both in programming and in mathematics, but means different
  things. Computational problems are described by mathematical
  functions.}.

\begin{takeaway}
  A \emph{computational problem} maps input data to output
  data. Mathematical functions clearly capture how these inputs map to
  the appropriate output.
\end{takeaway}

\subsection{Algorithms}

How do we solve a \emph{computational problem}? We need a procedure,
that is a ``recipe'' that we can follow blindly---just like a
machine---to get to the result. These recipes or procedures are
\emph{algorithms}: Sequences of instructions that solve a
computational problem.

\marginnote{
  \includestandalone[width=.8\marginparwidth]{images/addition.tikz}
  \captionof{figure}{Adding two natural numbers}
  \label{fig:school-addition}
}[-1.5cm]

Returning to the addition of two natural numbers, I have learned in
primary school an algorithm to do
that. Figure~\ref{fig:school-addition} shows the setup I would use to
add \numprint{4179} to 967. Here are the steps I would follow:
\begin{enumerate}
\item Write down the two given numbers in a grid and align their
  digits by significance: The least significant digit on the rightmost
  column. Keep a free row on top for possible carry-overs and
  another row below for the result. Keep an extra column on
  the left for a possible final carry-over.
\item Put your finger under the rightmost column.
\item If there is no digit to read, stop here.
\item Otherwise, read the digits in this column.
\item Add these digits to get their sum and the associated carry-over.
\item Write down this sum into the bottom cell of the current column.
\item Move your finger to the next column on the left.
\item If there is a carry-over, write it down in first cell.
\item Return to Step 3.
\end{enumerate}

This is our first algorithm: A recipe to add natural numbers! The
notion of algorithm is however not so well defined. I am not aware of
a single formal definition, upon which everyone agrees\sidenote{I
  found many attempts at a definition. See \cite{vardi2012,hill2016}
  or
  \href{https://en.wikipedia.org/wiki/Algorithm_characterizations}{algorithm
    characterizations} on Wikipedia for more details}. In this course,
I will reuse the definition given by D. Knuth in
TAOCP~\cite[p. 5--6]{knuth1978} (see Section~\ref{sec:resources}),
where he specifies the four following properties:
\begin{itemize}
\item An algorithm has \emph{inputs and outputs}. It consumes some data
  and produces some results. Our addition takes two natural numbers
  and outputs their sum.
\item An algorithm is \emph{finite}: It must terminate at some point
  and cannot have an infinite number of steps. Our addition of two
  numbers stops when we have added all pairs of digits.
\item An algorithm is \emph{well-defined}, and each step is
  non-ambiguous. In our addition, each step is about reading, adding,
  comparing or writing numbers. Children do not need to know how to
  add, they can use an addition table that gives both the result digit
  and the carry over as shown on Figure~\ref{fig:school-addition-table}.
  \marginnote{
    \includestandalone[width=.8\marginparwidth]{images/addition-table.tikz}
    \captionof{figure}{Addition table. Each cell contains two digit,
      the carryover and the sum.}
    \label{fig:school-addition-table}
  }[-1.5cm]
  
\item An algorithm is \emph{effective} and can be carried out by
  either a machine or human with pen and paper in a finite amount of
  time. Each step must be feasible. As for the addition, children add
  numbers this way on a daily basis, in a few minutes.
\end{itemize}

Do not confuse algorithm and computation. As for the addition, the
\emph{algorithm} is the list of steps to follow whereas the
\emph{computation} is what happen when a computer (a machine or a
child) goes through a particular addition.

\subsection{Data Structures}

An algorithm is a sequence of steps that manipulates data to solve a
problem. It necessarily produces and transforms data and needs a place
to store it---a sort of ``scratch pad'' if you will. This scratch pad
is the \emph{data structure}: How we organize the data our algorithm
manipulates.

In our addition example (see Figure~\ref{fig:school-addition}) we use
a ``grid'' that stores all the data, including the two
given numbers (the inputs), the result (output) and the carryovers
(intermediate results). This grid has four rows and one more column than
the longest given numbers has digits.

Many data structure are possible for a given algorithm, and an
appropriate data structure enables efficient algorithms. We will
discuss various schemes such as lists, trees, graphs, etc. Each has
its strengths and its weaknesses. As for our addition, we could have
written it down as a list of symbols:
\begin{equation}
\numprint{4179} + \numprint{967} = \numprint{5146}
\end{equation}

But that would have been harder. Primary school teachers use this grid
because it makes things easier for children: They
proceed---mechanically---by columns. Only when we become more fluent
do we get rid of the grid. The very same apply to algorithms:
Appropriate data structure is the key to their efficiency.

\begin{takeaway}
  An \emph{algorithm} is a \emph{finite} sequence of
  \emph{non-ambiguous} instructions, which processes its inputs to
  produce the solution of a \emph{computational problem}. To work
  efficiently, algorithms store their data into dedicated \emph{data
    structures}.
\end{takeaway}

\section{How to Describe an Algorithm?}

Once we have solved a computational problem, we have to communicate
our solution: Explain it to our colleagues or simply to ``program'' a
machine to do it. So, what is the best way to describe an
algorithm? If the computer is a person our bullet list of plain
English instructions may very well do the job. If the computer is a
machine however, there we will have a hard time to get the machine
understand all the nuances of our natural languages. Let us review a
few commonly used approaches: Natural languages, flowcharts,
pseudo-code and programs.

\paragraph{Using Natural Language}
The simplest way to describe an algorithm is to use plain English,
though this often lead to ambiguous text, which rules out the use
machines. This is what we used in the previous section for our
addition algorithm.

\paragraph{Using Flowcharts} Another human-friendly way is to use a
flowchart as shown on Figure~\ref{fig:flowchart}. In a flowchart, the
steps of an algorithm are shown as boxes connected by arrows. The
flowchart syntax distinguishes between various type of steps such as
processes, document, decisions, references, etc using different
shapes. Figure~\ref{fig:flowchart} only use ``processes'' (shown as
rectangles) and decisions (shown as diamond). A flowchart makes the
control structure (loops and decisions) very explicit, but the
graphical syntax is quite space consuming and may not scale to complex
algorithms.

\begin{figure}[htbp]
  \begin{center}
    \includestandalone[width=.9\textwidth]{images/flowchart.tikz}
  \end{center}
  \caption{The grade-school addition algorithm portrayed as a
    flowchart.}
  \label{fig:flowchart}
\end{figure}

\paragraph{Using Pseudocode} A third human-friendly solution is to use
\emph{pseudo-code}. The idea is to combine control structures common
in most imperative programming languages (loops and conditional) with
plain English or mathematical notation in order to express succinctly
the main idea of an algorithm. For our addition algorithm
we could write something like:
\begin{algorithm}
  \KwIn{$x$: Sequence of digits}
  \KwIn{$y$: Sequence of digits}
  \KwOut{$z$: Sequence of digits such as $z = x + y$}
  Setup $x$ and $y$ into a grid \;
  Place your finger under the right most column\;
  \While{there are digits in the column}{
    Read these digits\;
    Add them up to get the sum and the carry\;
    Write down this sum in the bottom cell\;
    Move your finger to the next column on the left\;
    Write down this carry in the top cell\;
  }
  \Return{the last row of the grid}\;
\end{algorithm}

\paragraph{Using a Program} All these representations help communicate
algorithms with people, so they all rely on natural language, which
may be ambiguous. We will see in the next Lecture how we can convert
pseudo code into machine code, that code that machine can understand.

\begin{takeaway}
  There is an direct relationship between the actions we stipulate in
  an algorithm and the capabilities of the computer we use to execute
  it.
\end{takeaway}


\section{Course Overview}

Are we going to go through a never-ending list of algorithms and
data-structure? In a sense yes, but bear in mind that the point of the
course is to practice solving problems. Algorithms and data
structures are just a means.

\subsection{Syllabus}

Figure~\ref{fig:syllabus} details the topics we will cover. My
strategy is to start with the concepts and theories and then to go
through the data structures and algorithms I think are the most
relevant. The two first weeks focus on the basics: What is an
algorithm, what is a program, how does a computer executes a program,
how to measure time and memory consumption? etc. Then we apply these
again and again on basic algorithms and data structures: Records,
arrays, lists, recursion and sorting.

\begin{figure}[htbp]
  \begin{center}
    \includestandalone[width=.9\textwidth]{images/syllabus.tikz}
  \end{center}
  \caption{Topic covered during the 13 Weeks}
  \label{fig:syllabus}
\end{figure}

Then, we push on to more advanced concepts, including hashing (Week
7), trees (Week 8, 9) and graphs (Week 10). These are all the data
structures we will study but be sure there are many more. In the
remainder we will explore two other topics where algorithms play a key
role: Combinatorial search and regular expressions. Finally we will
wrap up with looking at two other computation models: Parallel
computing and Quantum computing. The knowledge we got is very general
and apply there as well.

\begin{takeaway}
  This course is hard: Learning to solve computational problems
  efficiently takes practice.
\end{takeaway}

\paragraph{No Prerequisites} Anyone can join and succeed in this
course: There is no prerequisite. I will use some high-school maths
(functions, sets, probabilities, summations, boolean logic), which I
recall in a separate appendix. The plan is to explain everything we
need as we go. This is \emph{not} a maths class.

\paragraph{Why to Study This?}
As you may already sense, this is the foundations of Computer Science,
and in turn, of Software Engineering. Image processing, cryptography,
compilers, networks, artificial intelligence, and other ``branches''
of Computer Science all develop algorithms and data
structures. Consider image processing as an example: How to detect the
contour of a shape in a bitmap?  What data structure is the most
suitable to represent an image in memory? Which algorithm is the
fastest? Which consume the least memory? This course lay down the
foundations to discuss, evaluate and compare algorithms and data
structures.

Besides, as a software engineer, your daily work is to solve
computation problems! Say we have to sort the score of the top players
in your new mobile game, for instance. Shall we go for a \emph{quick
  sort} or a \emph{radix sort}?  Should we roll out something of our
own? It is critical to know what already exists and where we it shines
and where it falls short.


\subsection{Learning Outcomes}

How will this course contribute to your career? By
developing competencies, skills and knowledge.

\paragraph{Competence}
Again, the point is to develop a single competence: \emph{Solving
  computational problems}, by designing algorithms and programs that a
machine can execute. This is difficult. This is not something we learn
and recite: One must practice.

\paragraph{Skills}
As we practice, we will build three core skills:
\begin{enumerate}
\item Design algorithms and data structures tailored for a specific
  problem.
\item Argue for their correctness and be confident that the solution
  we come up with actually solves the problem at hand.
\item Compare the time and memory consumed by our algorithms to
  alternative solutions.
\end{enumerate}

\paragraph{Knowledge}
This course contains a lot of knowledge, especially in the form of
known problems and their accepted solutions, which combine algorithms
and data structures. But this course is not exhaustive at all! It is just
an introduction, whose aim is only to establish the shared vocabulary and
culture to be effective as a professional. We will cover the basics:
Arrays, lists, searching, sorting, trees, graphs, hashing, etc.

\begin{takeaway}
  The single objective of this course is to train you at solving
  computational problems. This is hard enough.
\end{takeaway}


\section{Practicalities}
Overall, we have three sessions a week during 13 weeks. Each session
is divided into 30 minutes of lectures and 1h30 of exercises or lab
session.

\paragraph {Lectures}
I broke down the content into 30 minutes long lectures. Each lecture
addresses a specific topic. I use various programming languages as
examples, avoiding those that are too verbose on slides. Keep in mind
that algorithms and data structures apply to any programming
language. I provide written lecture notes for the most important
lectures, so you can focus on understanding.

\paragraph {Lab sessions}
The lab sessions aim at putting concepts into practice. These are
either pen-and-paper exercises or programming tasks. Programming tasks
are in Java, but feel free to use any other language that makes things
easier for you. The point is to solve the programming tasks, not to
struggle with the programming language. There is no need to
know Java programming beforehand, but a prior exposure will help. I
provide selected written solutions, where I see fit.

\paragraph {Weekly Quizzes}
Every week, I will post a quiz to help you assess your understanding
of the week's topics. They do not account in your grade but please
reach out if any thing is unclear.

\paragraph{Examinations}
Your grade depends solely on the final examination. To get prepared as
well as possible, I include two ``home'' examinations around Week 4 and
8. These are optional, but should you feel the need for it, I can
grade them. All examinations follow the same pattern: 100 points
divided among four parts with 5 questions each.

\section{Expectations}

\paragraph{What do I expect from you?}

\emph{This content is hard}---I cannot stress this enough. Some of
which are non-trivial. Yet, I made everything optional, but the final
examination. I see this course is an opportunity to study algorithms
and data structures. Here is what I \emph{strongly} recommend:

\begin{itemize}
\item Attend the lecture physically. Even more, ask questions!
  Attending empower you to ask questions and to stop me if I am
  unclear. Please do so: I am happy to adjust or answer to the best of
  my knowledge.
\item Attend the lab sessions and go through the exercises. In my
  experience, exercises (whether on paper or on computer) is the only
  thing that really ground our understanding.
\item Go through the home examinations. I believe this is a good
  preparation to the final one and also a good way to see where we
  stand.
\item Reach out if you need any help.
\item Practice. practice, practice.
\end{itemize}

As you can see in Table~\ref{tab:syllabus}, this course is
``content-heavy''. I don't think one can master the whole thing in a
week or so before the final examination. A strongly recommend you
spread your work over the 13 weeks.

\paragraph {What can you expect from me?}

I am there three times a week for all lectures and lab sessions. Feel
free to come and ask questions. I try to be available and responsive,
by email especially\sidenote{I am not sitting in NTNU on a daily
  basis. Reaching out by email is the best way to get help.}. I try to
reply within 24 hours, but bear in mind that I am also a full time
engineer. Expect answers outside office hours.


\section{Additional Resources}
\label{sec:resources}

There are loads of books
\cite{atto1974,melhorn2008,levitin2011,weiss2014,skiena2020}, videos,
online courses, tutorials or blog posts on algorithms and data
structures. This course in an introduction: I am not presenting
anything new or innovative. Let me know if the angle I have chosen is
unclear or does not fit your background. We can look at other
resources where the same content is taken from a different
perspective.

\emph{I do \textbf{not} follow any textbook} in particular, but there
are a few ``references'' textbooks, whose authors are leading
authorities. Feel free to check them if you need a different treatment:

\begin{itemize}
\item ``The Art of Computer Programming'' (aka TAOCP) by Donald
  E. Knuth. This is \emph{the} reference. Five volumes so far:
  Fundamentals~\cite{knuth1978}, semi-numerical~\cite{knuth1997},
  sorting and searching~\cite{knuth1998}, combinato\-rial
  al\-go\-ri\-thms~\cite{knuth2011}. I found the treatment very detailed
  and oriented towards mathematicians. Beyond the scope
  of this course.
\item ``Introduction to Algorithms'' by Thomas H. Cormen et
  al~\cite{cormen2009} is another famous reference. It covers most of
  the topic I selected and many more, except maybe correctness.
\item ``Algorithms'' by Robert Sedgewick~\cite{sedgewick2014} is another
  well-known textbook, which is used in some very successful online
  courses. Does not cover the whole syllabus.
\item ``Data structures and algorithms in Java'' by Michael
  T. Goodrich and Roberto Tamassia~\cite{goodrich2014} fits very well this
  course. Most content we will cover is also treated there.
\end{itemize}

There are also online course on platforms such as
\href{https://www.coursera.org}{Coursera} or
\href{https://www.edx.org}{EdX} for instance. They can be a good
alternative to my presentations. Check however that the content they
cover matches the syllabus.

You will also find many online blog posts and tutorials such as
\href{www.geeksforgeeks.com}{GeeksForGeeks} or
\href{www.tutorialspoint.com}{TutorialsPoint}. Please keep a critical
eye as they may contain mistakes (as does this course).

Finally, there are many web sites such
\href{http://www.hackerrank.com}{HackerRank} or
\href{https://leetcode.com}{LeetCode} where one can practice solving
problems. Many of these problems are direct applications of the
concepts we will study. These web sites are great resources to gain
fluency.


\section*{Conclusions}
Hopefully, you have a better grasp at what this is about. Let us get
started! There is a lot or ground to cover. Please reach out if you
have any questions or if you find mistakes in the slides, the lectures
notes or the lab sessions.

\bibliographystyle{acm}
\bibliography{references}

\end{document}
