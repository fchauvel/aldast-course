\documentclass{aldast}


\documentType{Lecture Notes}
\documentNumber{3.1}
\title{Abstract Data Types}
\author{F. Chauvel}

\begin{document}

\maketitle

\begin{abstract}
  We shall look at data, which, as we saw, stands for anything that
  can be encoded using a finite set of symbols. So far we have only
  manipulated numbers, but we shall see here how data types help us
  tame other kinds of data such as floating point numbers, characters,
  as well as compounds such as records, arrays, etc. We will introduce
  abstract data types (ADT) as a means to describe these data types
  regardless of their underlying symbolic representation.
\end{abstract}


\section*{Introduction}

So far we have used only numbers, but in real-life, we need a little
more. We want text, images, colors, sounds, dates, 2D points, users
records, etc. In this lecture, we will look at data types, what they
are and how they can help use design and reason on algorithms.

Say we are task with converting an color image to grayscale. This
sounds pretty remote from what our RAM can do? How can we write an
algorithm that does that?

\begin{itemize}
  \item How to represent more than just numbers with machine's symbols?
  \item How to move beyond int, float, characters and define our own
    composite data types
  \item How to leverage data types in algorithms

\end{itemize}
 

\section{Data Types}

So far we have manipulated only integers, because our RAM use Arabic
digits. What if we need the machine to manipulate other type of data,
say an image for instance? Well, from the machine standpoint, there is
nothing to do: Machines only crunch symbols. At the programming
language level however, we need new \emph{data types}.

\marginpar{
  \includestandalone[width=.9\marginparwidth]{images/adt.tikz}
  \captionof{figure}{Data types: Their programming interface and
    their internal representation}
  \label{fig:adt}
} As shown on Figure~\ref{fig:adt}, a \emph{data type} defines two
things: A symbolic representation and a \emph{programming
  interface}. The symbolic representation decides how we use the
machine's symbols to represent the data of interest (the image in our
previous example). The programming interface includes all the
procedures we use to manipulate this new data type. For images these
could be resizing, cropping, blurring, switching to gray scales,
etc.

\begin{takeaway}
  A \emph{data type} is a set of sequences of machine symbols that
  adhere to a specific representation and which we manipulate using
  specific programming interface.
\end{takeaway}

Data types come up handy for programmers because they make
programs easier to understand. Consider the following Java program
that decides whether the user answered ``yes'' or ``no''. The keywords
``boolean'' and ``String'' gives hints about the intention.
\begin{minted}{java}
  boolean isYes(String anwser) {
    return Character.toUpperCase(answer.charAt(0)) == 'Y';
  }
\end{minted}

C, Pascal, Java and other \emph{statically-typed languages} require
programmers to mark variables and parameters with the appropriate data
type. This way, the compiler can detect earlier invalid expressions
such as \mintinline{java}{var x = 25 / 'a'}, where we attempt to
divide a number by a character. Typing can however become a hassle
when they are less obvious, and that is why \emph{dynamically-typed
  languages} like Python, Ruby or JavaScript do not enforce
type-correctness and let typing mistakes surface as runtime errors.

\subsection{Symbolic Data Representation}

\marginpar{
  \includestandalone[width=.9\marginparwidth]{images/encoding.tikz}
  \captionof{figure}{Encoding and decoding with a specific representation format}
  \label{fig:representation}
} The first step towards a new data type is to find a way to represent
data with the machine's symbols. Once we agreed on a representation
format, we can extend the I/O device to translate back and forth
between data and symbols.  Figure~\ref{fig:representation} shows how
pthe representation format decides both encoding and decoding.

\paragraph{Encoding Data into Symbols}

% There are many encodings possible
Encodings schemes define how we represent a given piece of data using
the machine's symbols (see Figure~\ref{fig:representation}). There are
many encodings you may have heard of, such as ASCII or Unicode for
characters, sign-and-magnitude or 2s-complement for integers, IEEE 754
for floating point values, PNG for images, etc.

Take ASCII for example, the American standard for information
interchange. ASCII was developed during the 60ies to represent text
for computers and telecommunications. With ASCII, each character
occupies 7~bits, so the ASCII format only accounts for $128=2^7$
character symbols. That is enough anyway to capture the Latin
alphabet, common punctuation symbols, a few math symbols and more. The
character 'A' (uppercase) is represented by the number 65 or (or 41 in
hexadecimal), 'B' by 66, 'C' by 67, etc. Lowercase letters come a bit
further down with 'a' encoded with 97, 'b' with 98, etc.

\paragraph{Decoding Data from Symbols}

The exact opposite of encoding---when we read data out of symbols---is
known as decoding. Returning to ASCII, decoding the four bytes
\texttt{4A-6F-68-6E} (i.e., 74, 111, 104, 110 as decimal numbers)
would be decoded as the text ``John'', as shown on
Figure~\ref{fig:representation}.

The key point about decoding is that we need to know the underlying
representation in advance. Given a sequence of symbols, one cannot say what
encoding it comes from. Take again the four bytes \texttt{4A-6F-68-6E} or
example, they can represent:

\definecolor{testColor}{HTML}{4A6F68}
\begin{itemize}
\item The natural number \numprint{1248815214} as a 32-bit integer value ;
\item The real number \numprint{3922459.5} as a 32-bit floating point value ;
\item The text ``John'' in ASCII ;
\item Some sort of \textcolor{testColor}{greenish color} as an RGBA color ;
\item etc.
\end{itemize}

\begin{note}{Data types in assembly code}
  For example, our RAM manipulates only numbers in base 10 using
  Arabic digits as symbols (0, 1, 2, 3, \ldots 9). As a result,
  numbers occurs in various roles: Some represent memory addresses
  (e.g., the \texttt{IP} register), some represent opcodes (e.g., 1
  denotes \texttt{LOAD}), and some represent actual numbers. A single
  sequence of symbols can have multiple interpretations, and in
  general---without more information---one cannot say what a bunch of
  symbols stands for. Take a single number, say 7 for example: We
  cannot say for sure if this is an address, an opcode, or just the
  value seven. This matters because if, by mistake, we use it in
  place of an opcode, the RAM would just halt.
\end{note}

The machine itself remains completely oblivious of such
encoding/decoding: It only transforms symbols. In our simplified RAM
architecture, encoding and decoding would take place in the I/O
device: It would convert specific representations and present
data accordingly to the user.

\begin{takeaway}
  A \emph{data structure} is the representation (i.e., the memory layout)
  chosen for a particular data type.
\end{takeaway}

\subsection{Programming Interface}

The second thing that characterizes a data type is its programming
interface: Procedures to manipulate the data.

Consider for the example the integer type, which comes predefined in
most programming languages (e.g., \texttt{int} in Java or C/C++). The
integer type comes with predefined operations that mirror the
arithmetic and logical operations that exists in mathematics
(addition, subtraction, division, modulo, comparisons, etc.), as well
as conversions to other data types such as floating point numbers or
string. Some of these operations are directly supported by the
underlying machine, such as the addition in our RAM, others require
dedicated procedures.

Character is another example. In Java, the associated programming
interface includes procedures such as \texttt{isLetter},
\texttt{isDigit}, \texttt{isWhiteSpace}, \texttt{isUpperCase},
\texttt{isLowerCase}, \texttt{toLowerCase}, and many more. These
procedures defines what one can do with a character, and in turn, what
a character is. As shown on Figure~\ref{fig:adt}, the programming
interface is all that matters to programmers, as one could possibly
change internal representation as long as the interface remains the
same. Think of programs that manipulate simple characters: The actual
representation (ASCII, Unicode, etc.) may be irrelevant.

Programming interface is what matters when it comes to algorithms and
for this course in particular.

\subsection{Composing Data Types}

\paragraph{Primitive Types}

High-level programming languages such as C, Java, Python natively
supports most common data types. The compilers (or interpreter) hides
the underlying representations and expose their programming interface
through keywords, operators or standard libraries. These
\emph{primitive} data types include
\begin{itemize}
\item Boolean values (true / false), which come along with
  conjunction (and), disjunction (or), and negation (not) operators.
\item Integer values, which support both arithmetic operations as
  well as comparisons.
\item Floating point values, which also support both arithmetic
  operations and comparisons
\item Characters, often encoded either in ASCII or in Unicode.
\item Bytes (8 bits), correspond to a raw sequence of symbols
\end{itemize}

\paragraph{Compound Types}

Primitive data types are programming languages give us to play with,
but we very often need to compose them in order to build new
``compound'' data types that capture domain concepts, such as color,
2D point, dates, time, user record, etc. We give here a brief
reminder, but refer to \cite[Chap. 7]{scott2009} for a more
comprehensive treatment. Programming languages provides three main
ways to compose data types, namely structures, arrays, and variants.

\paragraph{Records} (also known as structures or tuples) includes
multiple entries called \emph{fields}, each with its own data
type. Records pp resemble tuples in mathematics which results from the
Cartesian product over sets, such as $(x,y) \in T_1 \times T_2$. The
key point of records is to access fields using their name. In Pascal
for example, one could describe a player in game using the following
record type
  \begin{minted}{pascal}
    type Date = record
      name: string;
      score: integer;
      isCPU: boolean;
    end;
  \end{minted}

  Figure~\ref{fig:record} illustrate how the compiler may lay out the
  record fields in memory. The details vary from compiler to compiler,
  but the principle remains the same: Provided the record starts at
  the \emph{base address} $b$, the address of the k-th field is given
  by:

  \begin{equation}
    address(f_k) = b + \sum_{i=1}^{k} size(T_i) \label{eq:record}
  \end{equation}

  The compiler takes case of this and provides us with direct access
  to each fields by name in constant time: Offset to all fields are
  precomputed at compile time.
  
  \begin{figure}[htbp]
    \begin{center}
      \includestandalone[width=.9\textwidth]{images/records.tikz}
    \end{center}
    \caption{Memory layout for a record $T_1 \times T_2 \times T_3$}
    \label{fig:record}
  \end{figure}

  % Example of record, date, user, student (name, isExchange, score, player, isBot)
  
\paragraph{Arrays} represents a sequence of items, all from the same data
  type. Formally, we can think of arrays as function $a$ that maps
  integer to specific item such $a: \mathbb{N} \to T$. The key point
  of arrays it to access items using their position in the
  sequence. Lecture 3.2 will dive into arrays. In C for example,
  one could represent calendar dates using an array:
  \begin{minted}{c}
    typedef int date[3];
  \end{minted}
  
  Figure~\ref{fig:array} shows how an array containing 3 items of a
  type $T$ could be laid out in memory. The array is allocated from a
  \emph{base address} ($b$). Since arrays contains items of a single
  data type $T$ (whose size is known), we can deduce the address of
  $k$-th item using:
  \begin{equation}
    address(k) = b + k * size(T) \label{eq:array}
  \end{equation}
  
  This is actually the same as Equation~\ref{eq:record}, but for a
  single data type. Here as well, the compiler takes care of this and
  provides us access by index in constant time. This makes array the
  go-to data structure for \emph{random access}: When little is known
  on which item will be accessed most often.
  
  \begin{figure}[htbp]
    \begin{center}
      \includestandalone[width=.9\textwidth]{images/arrays.tikz}
    \end{center}
    \caption{Memory layout for an array of 3 elements of type $T$}
    \label{fig:array}
  \end{figure}
  
\paragraph{Variants} (also known as unions) represent a single field which
  can belong to multiple data types: It can be decoded using different
  format. Formally, a variant captures the union of multiple data
  types $T_1 \, \cup \, T_2$. We access data through the name we give
  to specific interpretation. For example, in C, we could write:
  \begin{minted}{c}
    union Number {
      int asInteger;
      float asFloat;
    };
  \end{minted}


Now, we can build whatever data type we please and we can combine them
using arrays, records or variants.

\section{Abstract Data Types}
\label{sec:adt}

Data types are what we need when we create a new programming language,
but from a pure algorithmic perspective, this is not what we need. The
internal representation is irrelevant. Ideally, we do not want to
write algorithms that only apply to ASCII characters. We want
algorithms that apply to characters in general, whether they are
encoded in ASCII, in Unicode. From an algorithmic perspective, the
representation is irrelevant, and what we should focus on is the
programming interface.

Unfortunately, the programming interface in itself is not enough to
define what \emph{any} character data type ought to offer. The
procedures relates to one another in very specific ways. For instance,
if I convert a character to upper case and then convert it back to
lower case, I would get back to the same original character. So to be
valid, a character data type has to offer a set procedures that behave
properly.

An \emph{abstract data type} (ADT)\sidenote{The concepts of
  \emph{information hiding}~\cite{parnas1972},
  \emph{encapsulation}~\cite{zilles1973} and ADTs~\cite{liskov1974}
  are the foundation of modern object-oriented languages.} defines the
programmer's expectations over a programming interface, irrespective
of its representation. Representation is an implementation detail, and
by hiding it from our algorithms, they become more generally
applicable.


\subsection{Defining an ADT}

An abstract data type defines three elements: a set of \emph{domains},
a set of \emph{operations} and a set of \emph{axioms}. Formally, an
ADT defines an algebra over the possible values of the data types. We
will illustrate each on a Boolean ADT.

\paragraph{The Domains} describe what the ADT is about, including all
the data types manipulated by the programming interface. The term
``domain'' comes from mathematics, where it stands for the set of values for
which a given function is defined. For our Boolean ADT, the domain
boil down to the set of boolean values $\mathbb{B}$.

\paragraph{The Operations} The operations are the procedures that we
can use to manipulate our ADT. They are often categorized into
\emph{constructor}, \emph{queries} and \emph{commands}. Constructors
instantiate the ADT from something else, queries convert our ADT into
something else, and commands modify it. The operations supported by
the Boolean ADT includes:
\begin{itemize}
\item Constructors
  \begin{itemize}
  \item $true: \varnothing \to \mathbb{B}$ creates $T$
  \item $false: \varnothing \to \mathbb{B}$ creates $F$
  \end{itemize}
\item Queries: None
\item Commands:
  \begin{itemize}
  \item $and: \mathbb{B} \times \mathbb{B} \to \mathbb{B}$ represents the conjunction of two Boolean values.
  \item $or: \mathbb{B} \times \mathbb{B} \to \mathbb{B}$ represents the disjunction of two Boolean values.
  \item $not: \mathbb{B} \to \mathbb{B}$ represents the negation.
  \end{itemize}
\end{itemize}

\paragraph{The Axioms} are the relationships between the procedures
that the ADT supports. In the case of the Boolean ADT, the axioms are:
\begin{enumerate}
\item $\forall x \in \mathbb{B}, \; and(x, false()) = false()$
\item $\forall x \in \mathbb{B}, \; and(x, true()) = x$
\item $not(true()) = false()$
\item $not(false()) = true()$
\item $\forall x,y \in \mathbb{B}^2, \; or(x, y) = not(and(not(x), not(y)))$
\end{enumerate}

\begin{takeaway}
  An abstract data type (ADT) captures the inherent relationships
  between the procedures that form the programming interface. It
  includes a domain, a set of operations, and a set of axioms that
  constrain the behavior of the procedures.
\end{takeaway}

\subsection{ADT and Correctness}

We touch upon correctness of algorithms in Lecture 1.3 and ADT is a
very convenient tool for that. Since ADT for the specification of a
data type, we can use them in two situations:
\begin{itemize}
\item Thinking about the correctness of an algorithm that uses our
  ADT. For example, if we have a program that contains the following
  boolean expression $or(true(), x)$, we can use the axioms of our
  Boolean ADT as follows:
  \begin{align*}
    \forall x \in \mathbb{B}, or(true(), x) &= not(and(not(true()), not(x)) \tag{Axiom 5} \\
                                            &= not(and(false(), not(x)) \tag{Axiom 3} \\
                                            &= not(false()) \tag{Axiom 1} \\
                                            &= true() \tag{Axiom 4}
  \end{align*}
  ADTs make explicit what we assume to be true when we design
  algorithms and thus greatly simplify reasoning about correctness.
\item Thinking about the correctness of an algorithm that implements
  the programming interface of our ADT. In this case, our axioms must
  be established: either proven or tested. Thinking in terms of ADT is
  one possible way to identify relevant test cases.
  
\item Thinking about the correctness of an implementation of our ADT.
\end{itemize}

\section{Conclusion}
We saw what is a data type, and how we can define new data types
either primitive data types that encode specific type of data into
machine symbols, or composite data types that recombines existing
ones using \emph{records}, \emph{arrays} or \emph{variants}.

We also saw how to abstract away the underlying machine representation
in order to define abstract data types. In the rest of this course, we
will explore well-known ADTs for sequence, sets, trees, etc. We will
look at alternative ``representations'' and see how and why they yield
different efficiency trade-offs.

\bibliographystyle{acm}
\bibliography{references}

\end{document}