% Created 2021-11-13 Sat 11:33
% Intended LaTeX compiler: pdflatex
\documentclass[11pt]{article}
\usepackage[utf8]{inputenc}
\usepackage[T1]{fontenc}
\usepackage{graphicx}
\usepackage{grffile}
\usepackage{longtable}
\usepackage{wrapfig}
\usepackage{rotating}
\usepackage[normalem]{ulem}
\usepackage{amsmath}
\usepackage{textcomp}
\usepackage{amssymb}
\usepackage{capt-of}
\usepackage{hyperref}
\usepackage{minted}
\author{NTNU IDATA 2302}
\date{Nov. 2021}
\title{Text Processing\\\medskip
\large Week 12 Quiz}
\hypersetup{
 pdfauthor={NTNU IDATA 2302},
 pdftitle={Text Processing},
 pdfkeywords={},
 pdfsubject={},
 pdfcreator={Emacs 27.2 (Org mode 9.4.4)}, 
 pdflang={English}}
\begin{document}

\maketitle


\section{Questions}
\label{sec:org050f686}

\begin{enumerate}
\item What is the runtime efficiency of finding a text fragment on a
larger text, using brute force? (\(n\) is the length of the text
and \(m\) the length of the pattern to find).
\begin{enumerate}
\item \(O(n^2)\)
\item \(O(n^m)\)
\item \(O(n\cdot m)\)
\item \(O(n + m)\)
\item \(O(n \log m)\)
\end{enumerate}

\item What is runtime efficiency of matching a text fragment in a
larger text using the KMP algorithm?
\begin{enumerate}
\item \(O(n^2)\)
\item \(O(n^m)\)
\item \(O(n\cdot m)\)
\item \(O(n + m)\)
\item \(O(n \log m)\)
\end{enumerate}

\item Does the following automaton recognizes the word "world"?
(accepting states are denoted with double circles).
\begin{center}
\includegraphics[width=7cm]{dfa.png}
\end{center}

\begin{enumerate}
\item Yes
\item No
\item One cannot say
\end{enumerate}

\item Is the following automaton deterministic?
\begin{center}
\includegraphics[width=8cm]{deterministic.png}
\end{center}
\begin{enumerate}
\item Yes
\item No
\item One cannot say
\end{enumerate}

\item What regular expression does the following automata capture?
\begin{center}
\includegraphics[width=7cm]{regex.png}
\end{center}
\begin{enumerate}
\item (abcd)*
\item (abc)?d
\item (abc)+d
\item (abc)*d
\item abc|d
\end{enumerate}

\item What other computing models are Turing-complete?
\begin{enumerate}
\item Random access machines (RAM)
\item Lambda calculus
\item Cellular automata
\item None of the above
\end{enumerate}

\item Which problems does the complexity class P include?
\begin{enumerate}
\item Those that can be solved in polynomial times
\item Those that can be solved in polynomial time by a regular Turing machine
\item Those that can be solved in polynomial space
\item Those that can be solved in polynomial space by a regular Turing machine
\item Those that can be solved in polynomial time and space by a regular Turing machine
\end{enumerate}

\item Which problems does the complexity class NP include?
\begin{enumerate}
\item Those that can be solved in non-polynomial times
\item Those that can be solved in polynomial time by a non-deterministic Turing machine
\item Those that can be solved in non-polynomial time by a regular Turing machine
\item Those that can be solved in non-polynomial space
\item Those that can be solved in non-polynomial time and space by a regular Turing machine
\end{enumerate}

\item What is an NP-complete problem?
\begin{enumerate}
\item It is as hard as any other problems in NP but no harder
\item It is at least as hard as any other problems in NP and possibly even harder
\item It is an NP problem whose which we know the complete set of solutions
\item None of the above
\end{enumerate}

\item Does P = NP?
\begin{enumerate}
\item Yes
\item No
\item One cannot say (we can't decide)
\item We do not know yet
\end{enumerate}
\end{enumerate}

\section{Solutions}
\label{sec:org9e182ac}

\begin{enumerate}
\item \textbf{Answer (c)}. Brute force runs in \(O(m \cdot n)\).

\item \textbf{Answer (d)}. KMP runs in \(O(m + n)\).

\item \textbf{Answer (b)} No. While the sequences of transitions matches the
letters of the word "world", the accepting state \(s3\) implies it
accepts only "wo", "worwo", "worworwo" (as a regular expression
it is "wo(rwo)*").

\item \textbf{Answer (b)} No, this automaton is not deterministic, because in
State \(s3\), we cannot determine in which set we should transition
on "c".

\item \textbf{Answer (d)} The automaton recognizes the regular expression
"abc*d". We can loop through abc, but the only accepting state is
after d.

\item \textbf{Answers (a), (b) and (c)}. All these computations models are
\emph{Turing-complete}. They can all emulate a Turing a machine and a
Turing machine can emulate them.

\item \textbf{Answer (b)} P includes all the problems that can be solved in polynomial time
by a \emph{regular} Turing machine.

\item \textbf{Answer (b)} NP includes all the problems that can be solved in
polynomial time by a \emph{non-deterministic} Turing machine.

\item \textbf{Answer (a)} It is equivalent to the hardest problems in NP, but
still in NP.

\item \textbf{Answer (d)} We do not know yet.
\end{enumerate}
\end{document}