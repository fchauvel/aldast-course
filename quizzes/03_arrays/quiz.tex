% Created 2021-09-11 Sat 06:15
% Intended LaTeX compiler: pdflatex
\documentclass[11pt]{article}
\usepackage[utf8]{inputenc}
\usepackage[T1]{fontenc}
\usepackage{graphicx}
\usepackage{grffile}
\usepackage{longtable}
\usepackage{wrapfig}
\usepackage{rotating}
\usepackage[normalem]{ulem}
\usepackage{amsmath}
\usepackage{textcomp}
\usepackage{amssymb}
\usepackage{capt-of}
\usepackage{hyperref}
\usepackage{minted}
\author{NTNU IDATA 2302}
\date{Sep. 10, 2021}
\title{Arrays\\\medskip
\large Week 3 Quiz}
\hypersetup{
 pdfauthor={NTNU IDATA 2302},
 pdftitle={Arrays},
 pdfkeywords={},
 pdfsubject={},
 pdfcreator={Emacs 27.2 (Org mode 9.4.4)}, 
 pdflang={English}}
\begin{document}

\maketitle


\section{Questions}
\label{sec:orgcf7c15f}

\begin{enumerate}
\item Can a record contain an arbitrary large number of fields?
\begin{enumerate}
\item Yes
\item No
\end{enumerate}

\item In an array, how to calculate the address of the i-th bucket?
\begin{enumerate}
\item \(\text{base address} + i\)
\item \(\text{base address} + i + \text{bucket size}\)
\item \(\text{base address} * (i + \text{bucket size})\)
\item \((\text{base address} * i) + \text{bucket size}\)
\item None of the above
\end{enumerate}

\item How does the time efficiency of accessing the i-th bucket vary in an
array?
\begin{enumerate}
\item It varies depending on the index of the array.
\item It remains constant regardless of the value of \texttt{i}.
\item It varies depending on the location of the base address.
\item None of the above.
\end{enumerate}

\item Why does the insertion (at an arbitrary position) take a linear
time?
\begin{enumerate}
\item Because we have to iterate through the array to find where to
insert.
\item Because we have to shift every subsequent bucket toward the
end.
\item Because accessing any bucket takes a linear time.
\item None of the above.
\end{enumerate}

\item For any array, which scenario takes the least runtime?
\begin{enumerate}
\item To delete the first bucket
\item To delete the last bucket
\item Both take the same amount of time
\end{enumerate}

\item What differentiate a dynamic array from a static one?
\begin{enumerate}
\item Dynamic arrays increase their capacity when they are full.
\item Dynamic arrays decrease their capacity when they become empty.
\item Dynamic arrays have a fixed capacity but their length varies.
\item None of the above.
\end{enumerate}

\item What distinguishes \emph{amortized analysis} from \emph{average
case analysis}?
\begin{enumerate}
\item Amortized analysis accounts for multiple operations on the same
data structure, whereas average case analysis characterizes a
single operation.
\item Average case analysis requires a probabilistic approach.
\item None of the above.
\end{enumerate}

\item Which of the following search algorithms can be used with any sorted
sequence?
\begin{enumerate}
\item Linear search
\item Jump search
\item Binary search
\item Interpolated search
\item Random search (we open buckets at random until we find what we are
looking for).
\item All of the above.
\end{enumerate}

\item Does the binary search always return a correct result?
\begin{enumerate}
\item Yes
\item No
\end{enumerate}

\item What best describes the worst case runtime efficiency of the binary
search?
\begin{enumerate}
\item \(O(n)\)
\item \(O(n^2)\)
\item \(O(n \cdot \log n)\)
\item \(O(\log n)\)
\item \(O(\sqrt{n})\)
\end{enumerate}
\end{enumerate}

\section{Solutions}
\label{sec:orgfe52edb}

\begin{enumerate}
\item \textbf{Answer (a)} Yes. There is no limit on how many fields a record
can have.

\item \textbf{Answer (c)} \(\text{base address} * (i + \text{bucket size})\)

\item \textbf{Answer (b)}, it remains constant regardless of the value of \texttt{i}.

\item \textbf{Answer (b)}. We must make room for the new bucket to be inserted
and we do that by shifting every subsequent bucket toward the
end.

\item \textbf{Answer (b)}. Deleting the last bucket takes constant time,
because there is no need to shift back all the subsequent buckets
toward the beginning.

\item \textbf{Answers (a) and (b)}. Dynamic arrays allocate more (or less)
memory when they become full (or empty).

\item \textbf{Answers (a) and (b)} Amortized analysis describes a sequence of
operation but does not requires a probabilistic approach. It
focuses on the average efficiency over multiple operations.

\item \textbf{Answers (a), (b), (c), (e) and (f)}. Only the interpolated search
does not apply on sorted arrays, because it not only requires the
array to be sorted, but also that the values be uniformly
distributed.

\item \textbf{Answer (a)} Yes. The binary search always returns a correct result, though
the desired value may not be in the array.

\item \textbf{Answer (d)} The tightest bound is \(O(\log n)\).
\end{enumerate}
\end{document}