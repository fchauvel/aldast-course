% Created 2021-09-15 Wed 22:33
% Intended LaTeX compiler: pdflatex
\documentclass[11pt]{article}
\usepackage[utf8]{inputenc}
\usepackage[T1]{fontenc}
\usepackage{graphicx}
\usepackage{grffile}
\usepackage{longtable}
\usepackage{wrapfig}
\usepackage{rotating}
\usepackage[normalem]{ulem}
\usepackage{amsmath}
\usepackage{textcomp}
\usepackage{amssymb}
\usepackage{capt-of}
\usepackage{hyperref}
\usepackage{minted}
\author{NTNU IDATA 2302}
\date{Sep. 15, 2021}
\title{Lists, Stacks \& Queues\\\medskip
\large Week 4 Quiz}
\hypersetup{
 pdfauthor={NTNU IDATA 2302},
 pdftitle={Lists, Stacks \& Queues},
 pdfkeywords={},
 pdfsubject={},
 pdfcreator={Emacs 27.2 (Org mode 9.4.4)}, 
 pdflang={English}}
\begin{document}

\maketitle


\section{Questions}
\label{sec:orgc2b68a2}

\begin{enumerate}
\item What is the worst case runtime efficiency of the selection sort?
\begin{enumerate}
\item \(\Theta(1)\)
\item \(\Theta(\log n)\)
\item \(\Theta(n)\)
\item \(\Theta(2^n)\)
\item None of the above
\end{enumerate}

\item In average, bubble sort runs faster than insertion sort.
\begin{enumerate}
\item Yes
\item No
\item One cannot say
\end{enumerate}

\item Sorting algorithms apply only on arrays.
\begin{enumerate}
\item True
\item False
\end{enumerate}

\item Linked list have no fixed capacity
\begin{enumerate}
\item True
\item False
\end{enumerate}

\item Why does the insertion in a linked list have a runtime in \(\Theta(n)\)?
\begin{enumerate}
\item Because one must shift all nodes towards the end of the list.
\item Because one must traverse the list until one find the nodes
where the insertion should take place.
\item Because the memory allocation also takes \(\Theta(n)\)
\end{enumerate}

\item What is the runtime efficiency of the \texttt{size} operation on linked-lists?
\begin{enumerate}
\item \(O(1)\)
\item \(O(n)\)
\item \(O(n^2)\)
\end{enumerate}

\item Linked-lists consume more memory than arrays.
\begin{enumerate}
\item Yes
\item No
\item One cannot say
\end{enumerate}

\item Stacks adhere to FIFO discipline (first-in, first-out)?
\begin{enumerate}
\item Yes
\item No
\end{enumerate}

\item Compared to an interface (in Java), what does an abstract data type
brings?
\begin{enumerate}
\item The implementation of operations (or methods).
\item Constraints and dependencies across operations
\item Nothing
\end{enumerate}

\item Queues and stacks are data structures.
\begin{enumerate}
\item Yes
\item No
\item One cannot say
\end{enumerate}
\end{enumerate}


\section{Solutions}
\label{sec:orgfd992e7}

\begin{enumerate}
\item \textbf{None of the above.} In the worst case, the selection
sort runs in \(\Theta(n^2)\).

\item \textbf{No}. Bubble sort and selection sort both run in \(\Theta(n^2)\) in
average.

\item \textbf{False} Sorting algorithms apply to linked list or array
but the very details of their implementation vary.

\item \textbf{True} Linked lists allocate new memory fragments during each
insertion and therefore are constantly resizing.

\item Because one must sequentially traverse the list until the nodes
where the insertion should take place. This takes \(O(n)\) runtime.

\item \textbf{Either \(O(1)\) or \(O(n)\)} In its simplest form, a linked-list
requires a traversal to compute how many nodes are linked together,
which yields a runtime in \(O(n)\). One can however improve this by
adding a \texttt{length} variable in the list itself and increment it on
each insertion. This would yield \(O(1)\).

\item \textbf{Yes} because we must store a pointer to the next record for each
item in the list.

\item \textbf{No}. Stacks implement a Last-in first-out discipline. It is queues
that implement the FIFO discipline.

\item Compared to an interface, an ADT adds constraints and dependencies
across operations.

\item \textbf{No} Queues and stacks are abstract data types, which can be
implemented by various data structures, including linked lists or
arrays.
\end{enumerate}
\end{document}