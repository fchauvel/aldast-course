% Created 2021-11-20 Sat 15:46
% Intended LaTeX compiler: pdflatex
\documentclass[11pt]{article}
\usepackage[utf8]{inputenc}
\usepackage[T1]{fontenc}
\usepackage{graphicx}
\usepackage{grffile}
\usepackage{longtable}
\usepackage{wrapfig}
\usepackage{rotating}
\usepackage[normalem]{ulem}
\usepackage{amsmath}
\usepackage{textcomp}
\usepackage{amssymb}
\usepackage{capt-of}
\usepackage{hyperref}
\usepackage{minted}
\author{NTNU IDATA 2302}
\date{Nov. 2021}
\title{Beyond Classical Computing\\\medskip
\large Week 13 Quiz}
\hypersetup{
 pdfauthor={NTNU IDATA 2302},
 pdftitle={Beyond Classical Computing},
 pdfkeywords={},
 pdfsubject={},
 pdfcreator={Emacs 27.2 (Org mode 9.4.4)}, 
 pdflang={English}}
\begin{document}

\maketitle


\section{Questions}
\label{sec:org78b28eb}

\begin{enumerate}
\item How many memories has a P-RAM machine?
\begin{enumerate}
\item a single memory
\item as many as it has processors
\item infinitely many
\item None of the above
\end{enumerate}

\item The \emph{work} of a parallel algorithm is
\begin{enumerate}
\item the sum of all the instructions executed by all the processors
\item the number of instructions in the program
\item the number of active processors
\item None of the above
\end{enumerate}

\item The \emph{cost} of a parallel algorithm is
\begin{enumerate}
\item the number of processors
\item the number of time steps the algorithm takes to complete
\item the total number of active processors over time
\item the total number of processor over time
\end{enumerate}

\item The speed-up of a parallel algorithm A contrasts
\begin{enumerate}
\item the execution time of A with 1 and many processors
\item the execution time of A with the execution time of the best
known serial algorithm, both running on a single processor.
\item the execution time of A with the execution time of the best
known serial algorithm, both running on multiple processors.
\item none of the above
\end{enumerate}

\item In Quantum Computing, when measuring a \emph{qubit}, its value 
\begin{enumerate}
\item vanishes
\item becomes either 0 or 1 without superposition
\item becomes negative
\item is superposed on top of its previous value
\end{enumerate}

\item Quantum Computing always outperforms "classical computing"?
\begin{enumerate}
\item Yes
\item No
\end{enumerate}

\item As opposed to classical boolean gates, quantum gates are
\begin{enumerate}
\item symmetric
\item can form circuits
\item reversible
\item none of the above
\end{enumerate}

\item The \emph{merge-sort} algorithm adheres to 
\begin{enumerate}
\item a brute-force approach
\item a divide-and-conquer approach
\item a greedy approach
\item a randomized approach
\item None of the above
\end{enumerate}

\item The "Jump-search" algorithm adheres to 
\begin{enumerate}
\item a brute-force approach
\item a divide-and-conquer approach
\item a greedy approach
\item a randomized approach
\item none of the above
\end{enumerate}

\item What is the purpose of "screening" during the recruitment of
"programmer" candidates?
\begin{enumerate}
\item To select only the best of the best programmers
\item Find the most challenging problems
\item To reject people without sufficient programming abilities
\item None of the above
\end{enumerate}
\end{enumerate}

\section{Solutions}
\label{sec:orge9678ff}

\begin{enumerate}
\item \textbf{Answer (a)} P-RAM relies on a single memory, shared by an infinity
of processors.

\item \textbf{Answer (a)} The \emph{work} reflects the sum of all the instructions
performed by all processors. If an instruction is simultaneously by
two processors, it is counted twice.

\item \textbf{Answer (d)} The \emph{cost} reflects the number of processors required
during the whole execution of the algorithm (including active and
passive ones).

\item \textbf{Answer (a)} The \emph{speed-up} reflects the difference between running
the algorithm A on one processor and on many processors.

\item \textbf{Answer (b)} When measuring a qubit, its value looses any
superposition of states and becomes either 0 or 1 (but not both
anymore).

\item \textbf{Answer (b)} No. There are problems for which Quantum computing
is no faster than classical computing.

\item \textbf{Answer (c)} Quantum gates are reversible, which means the can be
used both ways: To generates outputs from inputs but also to
generate inputs from outputs.

\item \textbf{Answer (b)} \emph{merge-sort} follows a \emph{divide-and-conquer} approach.

\item \textbf{Answer (b)} I would say that jump-search also adheres to a
divide-and-conquer approach, since it tries to break down the
search into a list of segments and then uses linear search to explore
the selected segment.

\item \textbf{Answer (c)} Screening is there mainly to sort out those that are
actually able to write code from those who are not.
\end{enumerate}
\end{document}