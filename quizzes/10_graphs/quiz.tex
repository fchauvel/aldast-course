% Created 2021-10-30 Sat 21:14
% Intended LaTeX compiler: pdflatex
\documentclass[11pt]{article}
\usepackage[utf8]{inputenc}
\usepackage[T1]{fontenc}
\usepackage{graphicx}
\usepackage{grffile}
\usepackage{longtable}
\usepackage{wrapfig}
\usepackage{rotating}
\usepackage[normalem]{ulem}
\usepackage{amsmath}
\usepackage{textcomp}
\usepackage{amssymb}
\usepackage{capt-of}
\usepackage{hyperref}
\usepackage{minted}
\author{NTNU IDATA 2302}
\date{Oct. 2021}
\title{Graphs\\\medskip
\large Week 10 Quiz}
\hypersetup{
 pdfauthor={NTNU IDATA 2302},
 pdftitle={Graphs},
 pdfkeywords={},
 pdfsubject={},
 pdfcreator={Emacs 27.2 (Org mode 9.4.4)}, 
 pdflang={English}}
\begin{document}

\maketitle

\section{Questions}
\label{sec:org8e46919}

\begin{enumerate}
\item How many nodes are in the following graph?
\begin{center}
\includegraphics[width=.9\linewidth]{graph.png}
\end{center}

\item What type of graph is shown on the previous picture?
\begin{enumerate}
\item An undirected graph
\item A directed graph
\item A weighted graph
\item A network flow
\item None of the above
\end{enumerate}

\item Does the following matrix \(M\) accurately capture the previous
graph?
\begin{align*}
     M =  \begin{bmatrix}
        % A   B   C   D   E   F
          0 & 1 & 1 & 0 & 0 & 0 \\
          0 & 0 & 0 & 1 & 0 & 0 \\
          0 & 1 & 0 & 0 & 0 & 0 \\
          0 & 0 & 0 & 0 & 1 & 0 \\
          0 & 1 & 0 & 0 & 0 & 1 \\
          0 & 0 & 0 & 1 & 0 & 0 \\
          \end{bmatrix}
\end{align*}

\item How many \emph{simple paths} are there from Node D to Node A?
\begin{enumerate}
\item None
\item 1 path
\item 2 paths
\item 3 paths
\item None of the above
\end{enumerate}

\item How many components are in this graph?
\begin{enumerate}
\item 1
\item 0
\item \(2^6 = 64\)
\item \(6^2\)
\item None of the above
\end{enumerate}

\item Starting from Node A, in which order would a \emph{pre-order
depth-first traversal} reach all the nodes, provided that it
processes successors in alphabetical order?
\begin{enumerate}
\item A, B, D, E, F, C
\item A, C, B, D, E, F
\item A, B, C, D, E, F
\item None of the above
\end{enumerate}

\item Starting from Node A, in which order would a \emph{pre-order
breadth-first traversal} reach all the nodes, provided that it
processes successors in alphabetical order?
\begin{enumerate}
\item A, B, D, E, F, C
\item A, C, B, D, E, F
\item A, B, C, D, E, F
\item None of the above
\end{enumerate}

\item Dijsktra's algorithm finds
\begin{enumerate}
\item the shortest path between two nodes
\item the shortest paths between one given node and every others
\item the shortest paths between every possible pairs of node
\end{enumerate}

\item A shortest path may exist only when the graph
\begin{enumerate}
\item contains no paths whose total weight is negative
\item contains no circuits
\item contains no negative weights
\item contains no circuit with negative weights
\item contains no circuit whose total weight is negative
\end{enumerate}

\item A spanning tree is
\begin{enumerate}
\item a subgraph that includes all the vertices
\item a subgraph that includes all the edges
\item a subgraph that includes all the vertices but forms a tree
\item a subgraph that includes all the edges but forms a tree
\item None of the above
\end{enumerate}
\end{enumerate}


\section{Solutions}
\label{sec:org906e500}

\begin{enumerate}
\item \textbf{6}. This graph contains six nodes, namely A, B, C. D, E, and F.

\item \textbf{A directed graph}. This graph is directed because edges go in a
specific direction. For example, it is possible to go from A to
C, but not from C to A.

\item \textbf{No} The matrix \(M\) lacks at least the \emph{loop} that is, the
circuit from B to B.

\item \textbf{None} There is no path from D to A because edges are directed.
One can only navigate as indicated by the arrows.

\item \textbf{One} A component is a subgraph that is isolated from some nodes in
the graph. No vertex is isolated here, so this graph has a single
component.

\item \textbf{A, B, D, E, F, C}. Since the traversal is \emph{pre-order} and
\emph{depth-first}, the traversal starts on node A. 'A' gets printed
before its children because it is a pre-order traversal. Then,
children are pulled in alphabetical order, so B comes before
C. Finally, it goes \emph{depth first} and explore the first path all
the way until it reaches a dead-end.

\item \textbf{A, B, C, D, E, F}. Here we are talking about a breadth-first
traversal that reaches nodes by level: First come the nodes that
are one edge away from the source, then the nodes two edges away,
and so one and so forth.

\item \textbf{Answer (b)} Dijsktra's algorithm finds the shortest paths between a
given source node and all the other vertices.

\item \textbf{Answer (f)} A shortest path can exist only in a graph where
there is no circuit whose total weight is negative.

\item \textbf{Answer (c)} A spanning tree is a subgraph that includes all the
vertices and forms a tree.
\end{enumerate}
\end{document}