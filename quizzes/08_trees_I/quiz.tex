% Created 2021-10-16 Sat 22:19
% Intended LaTeX compiler: pdflatex
\documentclass[11pt]{article}
\usepackage[utf8]{inputenc}
\usepackage[T1]{fontenc}
\usepackage{graphicx}
\usepackage{grffile}
\usepackage{longtable}
\usepackage{wrapfig}
\usepackage{rotating}
\usepackage[normalem]{ulem}
\usepackage{amsmath}
\usepackage{textcomp}
\usepackage{amssymb}
\usepackage{capt-of}
\usepackage{hyperref}
\usepackage{minted}
\author{NTNU IDATA 2302}
\date{Oct. 16, 2021}
\title{Trees\\\medskip
\large Week  Quiz}
\hypersetup{
 pdfauthor={NTNU IDATA 2302},
 pdftitle={Trees},
 pdfkeywords={},
 pdfsubject={},
 pdfcreator={Emacs 27.2 (Org mode 9.4.4)}, 
 pdflang={English}}
\begin{document}

\maketitle

\section{Questions}
\label{sec:orgd1d78e4}

\begin{enumerate}
\item In a tree, the \emph{descendants} of the root include all the nodes of
the tree?
\begin{enumerate}
\item Yes
\item No
\end{enumerate}

\item The \emph{depth} of a node \(n\) denotes
\begin{enumerate}
\item The number of ancestors of \(n\)
\item The number of descendants of \(n\)
\item The number of siblings of \(n\)
\item None of the above
\end{enumerate}

\item The \emph{degree} of a node denotes
\begin{enumerate}
\item The number of ancestors of \(n\)
\item The number of descendants of \(n\)
\item The number of siblings of \(n\)
\item None of the above
\end{enumerate}

\item In a \emph{ternary} tree, each node can have
\begin{enumerate}
\item No child
\item One child
\item Two children
\item Three children
\item Four children
\end{enumerate}

\item What best describes the average runtime efficiency of inserting an
item in a binary search tree?
\begin{enumerate}
\item \(O(1)\)
\item \(O(\log n)\)
\item \(O(n)\)
\item \(O(n \log n)\)
\item \(O(n^2)\)
\end{enumerate}

\item In a binary search tree the successor of a node that has two
children is:
\begin{enumerate}
\item The maximum of the left subtree
\item The minimum of the left subtree
\item The maximum of the right subtree
\item The minimum of the right subtree
\item None of the above
\end{enumerate}

\item The successor operation in a binary search tree always succeeds?
\begin{enumerate}
\item Yes.
\item No.
\end{enumerate}

\item Binary search trees degenerate into linked lists when
\begin{enumerate}
\item Items are inserted in ascending order
\item Items are inserted in descending order
\item Items are inserted randomly
\item None of the above
\end{enumerate}

\item How does an AVL tree define the balance factor of a node?
\begin{enumerate}
\item \((height(left) + height(right)) / 2\)
\item \(height(left) - height(right)\)
\item \(\max(height(left), height(right))\)
\item \(\min(height(left), height(right))\)
\item None of the above
\end{enumerate}

\item Which of the following operations require a possible re-balancing
the AVL tree?
\begin{enumerate}
\item Insertion
\item Minimum
\item Maximum
\item Search
\item Successor
\item Predecessor
\item Deletion
\end{enumerate}
\end{enumerate}

\section{Solutions}
\label{sec:org57ae384}

\begin{enumerate}
\item \textbf{No} the descendants of the root node include all the node but the root.

\item \textbf{The number of ancestors of the node}. The depth of a node denotes
the number of links that separate the node from the root, which
is the same as the number of ancestors.

\item \textbf{The number of siblings} is named the \emph{degree} of a node.

\item \textbf{0, 1, 2, or 3} In a \emph{ternary-tree} each node can have at most
three children. Nodes may have less than three children though.

\item \textbf{\(O(\log n)\)} Insertion in a binary search tree boils down to
finding where to insert, which is comparable to a binary search,
and thus, it runs in \(O(\log n)\).

\item \textbf{The minimum of the right subtree}. The successor of a node with
children is the left-most node in the right subtree.

\item \textbf{No}. Searching for a successor may fail if the given item is not
present in the tree.

\item \textbf{(a) and (b)} When items are inserted in ascending order, the
right subtree degenerates into a list. By contrast, insertions in
descending order turn the left subtree into a list.

\item \textbf{\(height(left) - height(right)\)}

\item \textbf{Insertion, Deletion}, More generally any operation that changes
the content of the tree may require re-balancing it.
\end{enumerate}
\end{document}