% Created 2021-09-05 Sun 18:54
% Intended LaTeX compiler: pdflatex
\documentclass[11pt]{article}
\usepackage[utf8]{inputenc}
\usepackage[T1]{fontenc}
\usepackage{graphicx}
\usepackage{grffile}
\usepackage{longtable}
\usepackage{wrapfig}
\usepackage{rotating}
\usepackage[normalem]{ulem}
\usepackage{amsmath}
\usepackage{textcomp}
\usepackage{amssymb}
\usepackage{capt-of}
\usepackage{hyperref}
\usepackage[cache=false]{minted}
\author{Franck Chauvel}
\date{Sep. 5, 2021}
\title{Linear Search\\\medskip
\large Average Case Analysis}
\hypersetup{
 pdfauthor={Franck Chauvel},
 pdftitle={Linear Search},
 pdfkeywords={},
 pdfsubject={},
 pdfcreator={Emacs 27.2 (Org mode 9.4.4)}, 
 pdflang={English}}
\begin{document}

\maketitle
\tableofcontents


\section{Introduction}
\label{sec:org468ffc8}

In Lecture 3.1, we introduced the linear search algorithm, which
finds the index of the first bucket that contains a given value. It
traverses the array from the first bucket to the last. In this
lecture, I take the opportunity to review and apply the concept of
\emph{average case analysis}.

Recall how linear search works. We start looking at the first bucket
and we check the value it contains. If it is the value are looking
for, we returned the index and we are done. Otherwise, we move to
next bucket and look there. If the desired value is still not there,
we move the third bucket, and so on and so forth until we either
find the desired value or we reach the end of the array, in which
case we return -1 to indicate that we could not find the desired
value. Listing \ref{orge254781} details a possible implementation
in C.

\begin{listing}[htbp]
\begin{minted}[linenos,firstnumber=1]{c}
#include <stdio.h>

#define CAPACITY 20

int length = 5;
int array[CAPACITY] = {5, 4, 3, 2, 1};

int search(int value) {
  for (int index = 0; index < length ; index++) {
      if (array[index] == value) {
          return index;
      }
  }
  return -1;
}

int main(int argc, char** argv) {
  printf("Found %d at: %d\n", 4, search(4));
  printf("Found %d at: %d\n", 12, search(12));
}
\end{minted}
\caption{\label{orge254781}The "linear search" algorithm for arrays}
\end{listing}

Note that in C, there is no easy way to access the capacity nor the
length of the array, so these must be stored in separate variables.
Here I used global variables for the sake of simplicity, but global
variable should be avoided in practice. Note also that in Java, the
\texttt{length} stands for the capacity of the array, and its length is not
available.

\begin{verbatim}
Found 4 at: 1
Found 12 at: -1
\end{verbatim}

\section{Problem size}
\label{sec:orgafee275}

Let's find the runtime complexity of this "linear search", shown
by Listing \ref{orge254781}. Let's skip the memory analysis and
focus instead on the runtime since the memory consumption is
constant: We only use 1 single variable, namely \texttt{index}.

As before, the first question to ask is "What governs the size of
the problem"? Here, it is the length of the array. The larger is the
given array, the more buckets we have to look and check. Note that
the array and its length are not input parameter of our algorithm
per se, but they are accessible nonetheless as a global variable,
and dictates the size of the problem.

We don't want to return to the RAM machine and write some assembly
code. So the first step is to describe a cost model to make
explicit what we are counting.

Here, I suggest to count comparisons, that is the number of time we
execute Line 10. This is the only logic operation aside of
incrementing the index variable (cf. Line 9). It keeps the
calculation simpler, and the difference vanishes once we classify it
using the Big-Oh notation. We are ready to analysis the best, worse
and average cases.

\section{Best Case}
\label{sec:orgadb5753}

Let's first look at the best case. It is the scenario where, given
an array and its length and capacity, our algorithm is the
fastest. The earlier we found the desired value, the earlier we exit
the loop (see Line 11), and the shorter is the search. What
happens is that we check the first bucket (with index 0), and we
found the value there (see Line 10). So have made one single
comparison. Without any algebra, we can compute that our best case
of the order of \(n\), that is \(t \in \Theta(n)\). Nice and
sweet. Let's continue with the worse case.

\section{Worse Case}
\label{sec:orgb55d96b}

What input makes our algorithm run the longest time? It is when we
must check every single bucket to find the value, in which case we
make as many comparison as there are buckets. That is our worse case
is of the order of \(n\), or formally, \(t\in \Theta(n)\). Not however
that this worse case occurs in two situation: When the desired
value is in the last bucket, but also when the desired value if
not in the array. In that case, we must still check every single
bucket but we found nothing and exit Line 14.

\section{Average Case}
\label{sec:orga5c3294}

Finally, let's look at the average case. This is more involved,
but it's a good opportunity to apply algorithm analysis. Let's
take it informally first. Intuitively the average efficiency, if
the sum of efficiency for all possible cases divided by the number
of cases. So what are the possible cases?  There are \(n+1\). The
desired value can be in any of the bucket (that's already \(n\)
cases), or it is not in the array (that's another case). So what
is the runtime if the desired value is in the first bucket? Well,
in that case, we do one single comparison, that's the value we are
looking for and we return the current index. If the desired value
is in the second bucket, we check the first one, it is not there,
we check the second, it is there. We have made two
comparisons. The same holds if the desired value is the last
bucket: We check all the buckets one by one, and we found the
value in the last one. That's \(n\) comparisons. Finally, if the
value is not in any bucket, then we have checked all of them in
vain, and that also \(n\) comparison. So if there were 3 items in
our array, the average complexity would be \(t(n) =
  \frac{1+2+3+3}{4} = 2.25\).

Let us formalize that. Let us first define a random variable B (for
bucket), whose value indicates in which bucket the desired value
lies.
\begin{itemize}
\item the random variable \(B\) ranges from \(-1\) to \(n-1\). -1 indicates
that the desired value is not in the array, whereas other values
indicates the index of the bucket that contains it. That's \(n+1\)
values.
\item I assume a uniform probability distribution for the sake of
generality. I denote by P(B=i) the probability that B=i, that is
the probability that the desired value lies in the i-th
bucket. This probability remains constant, regardless of the
value of \(B\), that is \(\Pr[B=i] = \frac{1}{n+1}\).
\end{itemize}

We can now express the runtime as a function of both the input
size \(n\), and the random variable \(B\) as follows:

\[
  \text{time}(n,B) = \begin{cases}
      B+1 &  \text{when } 0 \leq B \leq n-1 \\
      n &  \text{when } B={-1} \\
      \end{cases}
  \]

With this definition the average runtime is the expected value
\(\text{time}(n. B)\), which we can calculate as follows:

\begin{align*}
  E[t(n,B)] & = \sum_{i=0}^{n-1}{ \big[\Pr[B\!=\!i] \cdot \text{time}(n,B) \big]} + \big[ \Pr[B\!=\!{-1}] \cdot t(n,-1) \big] \\
            & = \sum_{i=0}^{n-1}{ \big[ \frac{1}{n+1} \cdot (i+1) \big]}+ \left[ n \cdot \frac{1}{n+1} \right] \\
            & = \left[ \frac{1}{n+1} \cdot \sum_{i=0}^{n-1}{(i+1)} \right] + \frac{n}{n+1} \\
            & = \frac{1}{n+1} \cdot \left[ \sum_{i=0}^{n-1}{i} + \sum_{i=0}^{n-1} 1 \right] + \frac{n}{n+1} \\
            & = \frac{1}{n+1} \cdot \left[ \frac{(n-1)[(n-1)+1]}{2} + n \right] + \frac{n}{n+1} \\
            & = \frac{1}{n+1} \cdot \left[ \frac{n(n-1)}{2} + n \right] + \frac{n}{n+1} \\
            & = \frac{1}{n+1} \cdot \left[ \frac{n(n-1)}{2} + \frac{2n}{2} \right] + \frac{n}{n+1} \\
            & = \frac{1}{n+1} \cdot \frac{n(n-1) + 2n}{2} + \frac{n}{n+1} \\
            & = \left[ \frac{1}{n+1} \cdot \frac{n^2+n}{2} \right] + \frac{n}{n+1} \\
            & = \frac{n^2+n}{2(n+1)} + \frac{n}{n+1} \\
            & = \frac{n^2+n}{2(n+1)} + \frac{2n}{2(n+1)} \\
  E[t(n,B)] & = \frac{n^2 + 3n}{2(n+1)} \\
\end{align*}

Quick sanity check before we continue: We see that \(E[t(3,B)] =
  2.25\) as we found previously intuitively. Figure \ref{fig:org564412c}
show visually the runtime of the best, average and worse case of
the linear search.

\begin{figure}[htbp]
\centering
\includegraphics[width=.9\linewidth]{linear_search_runtime.pdf}
\caption{\label{fig:org564412c}Visualization of the average time-complexity of the linear search algorithm, shown in Listing \ref{orge254781}}
\end{figure}

Now we have found our formula for the average scenario. Let's find
an approximate upper bound. So, following the definitions in
\href{orders\_of\_growth.org}{orders of growth (Chapter 3)}, we must find a function, \(g(n)\), a
constant \(c\), and a constant \(k\), such as the product \(c \cdot
  g(n) \geq t(n, B)\), for all \(n \geq k\). As a first guess, I assume
that \(g(n) = n\) and that \(c=2\): Let's see where does that take us.
\begin{align*}
   c \cdot g(n) & \geq t(n, B) \\
   2n & \geq \frac{n^2 + 3n}{2(n+1)} \\
   4n & \geq \frac{n^2 +3n}{n+1} \\
   4n (n+1) & \geq n^2 +3n \\
   4n^2 + 4n & \geq n^2 +3n \\
   4n^2 - n^2 + 4n - 3n & \geq 0 \\
   3n^2 + n & \geq 0 \\
\end{align*}

This second-degree inequality \(3n^2 + n \geq 0\) holds
regardless of \(n\), so we can pick \(k\) as we please. So we have
shown that our average time-efficiency mode admits an upper bound
of linear order: \(t \in O(n), \forall \; k \geq 0\).

One down, one to go. Let's now turn to the lower bound. Again,
refer to the definition given in \href{orders\_of\_growth.org}{the previous lecture}. To find an
approximate lower bound, we have to find a function \(g(n)\), and
constant \(c\), and a constant \(k\), such as the product \$ c\(\cdot\)
g(n)\$ is lower than or equal to \(t(n, B)\) for all \(n \geq
  k\). Again, as a first guess, I assume that \(g(n) = n\), and that
\(c=\frac{1}{2}\). Let see what we get:

\begin{align*}
   c \cdot g(n) & \leq t(n, B) \\
   \frac{n}{2} & \leq \frac{n^2 + 3n}{2(n+1)} \\
   n & \leq \frac{n^2 +3n}{n+1} \\
   n (n+1) & \leq n^2 +3n \\
   n^2 + n & \leq n^2 + 3n \\
   0 & \leq n^2 - n^2 + 3n - n   \\
   0 & \leq 2n \\
   0 & \leq n \\
\end{align*}

This gives use a value for the constant \(k\). So we have shown that
our runtime model accepts and linear lower bound, that is \(t \in
  \Omega(n), \forall \; k \geq 0\).

We can conclude that our runtime is of the order of \(g(n) = n\),
because for any \(k \geq 0\), our models admits both a linear upper
bound and a linear lower bound, that is, \(t \in \Theta(n)\).
\end{document}