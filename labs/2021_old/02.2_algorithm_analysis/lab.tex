% Created 2021-09-02 Thu 05:51
% Intended LaTeX compiler: pdflatex
\documentclass[11pt]{article}
\usepackage[utf8]{inputenc}
\usepackage[T1]{fontenc}
\usepackage{graphicx}
\usepackage{grffile}
\usepackage{longtable}
\usepackage{wrapfig}
\usepackage{rotating}
\usepackage[normalem]{ulem}
\usepackage{amsmath}
\usepackage{textcomp}
\usepackage{amssymb}
\usepackage{capt-of}
\usepackage{hyperref}
\author{NTNU IDATA 2302}
\date{Aug 30, 2021}
\title{Algorithm Analysis\\\medskip
\large Lab Session 2.2}
\hypersetup{
 pdfauthor={NTNU IDATA 2302},
 pdftitle={Algorithm Analysis},
 pdfkeywords={},
 pdfsubject={},
 pdfcreator={Emacs 27.2 (Org mode 9.4.4)}, 
 pdflang={English}}
\begin{document}

\maketitle
\tableofcontents

\begin{abstract}
In this lab session, we look at several programs and we model how
their efficiency varies according to their input size. We will look at
the maximum between two numbers, a rock-paper-scissor game, and a
programs that draws a triangle on the terminal. In each case, we will
focus on the runtime efficiency, since memory efficiency does not
bring much hindsight for such short programs.
\end{abstract}


\section{Maximum of Two Numbers}
\label{sec:orgb501f4c}

Let us first look at a simple program that finds the maximum between
two given integers. Figure \ref{org1bd9571} shows a possible Java
implementation.

\begin{verbatim}
 1  public class Maximum {
 2  
 3      public static int maximum(int left, int right) {
 4          if (left < right) {
 5              return right;
 6          }
 7          return left;
 8      }
 9  
10      public static void main(String[] args) {  
11          System.out.println(maximum(10, 5));
12      }
13  
14  }
\end{verbatim}
\captionof{figure}{\label{org1bd9571}Selecting the maximum betwenen two given numbers}

Finding out the worst case (with respect to runtime) boils down to
finding the input that maximize the time spent by the underlying
machine running our program.

Because I assume a random access machine and a cost model where every
instruction takes 1 unit of time, the worst case is thus finding the
input that maximizes the number of instructions executed. In this case
there are only two possible "execution paths":

\begin{enumerate}
\item Either \texttt{left} is strictly less than \texttt{right}, in which case the program
directly returns \texttt{right} (see Line 5).

\item The \texttt{left} variable is greater or equal to \texttt{right}, in which case
our function returns \texttt{left} (see Line 7).
\end{enumerate}

\subsection{Explicit Conversion to RAM assembly}
\label{sec:orgc335fc0}

To compute how many instructions the machine execute in each case, one
could---in principle---convert it to a RAM assembly program and count
how many time each instruction is executed. Figure \ref{org9856654}
gives a possible assembly program.

\begin{verbatim}
 1  segment: data
 2        left     1        0           ; left takes 1 cell, set to 0
 3        right    1        0           
 4        maximum  1        0
 5  
 6  segment: code
 7                 READ     right       ; left := user-input
 8                 READ     left        ; right := user-input 
 9                 LOAD     0
10                 ADD      left
11                 SUBTRACT right
12                 JUMP     lmax        ; Go to 'lmax' if left >= right
13                 LOAD     0
14                 ADD      right
15                 STORE    maximum     ; maximum := right
16                 LOAD     0
17                 JUMP     done        ; Go to done
18        lmax:    LOAD     0
19                 ADD      left
20                 STORE    maximum     ; maximum := left
21        done:    PRINT    maximum     ; print(maximum)
22                 HALT     -1          ; stop the machine
\end{verbatim}
\captionof{figure}{\label{org9856654}A possible RAM assembly implementation of our Java maximum function}

The only complexity is how we compare left and right. We have only a
single JUMP instruction, which jumps only when the \texttt{ACC} register is
positive or null. To jump when left is greater, we must first place in
\texttt{ACC} the difference between \texttt{left} and \texttt{right}. Here, the two
execution paths we have identified earlier surface as follows:

\begin{enumerate}
\item If \texttt{left} is strictly less than \texttt{right}, then the machine would
start at Line 7, proceed until the \texttt{JUMP} instruction
(Line 12). At this point, the \texttt{ACC} register contains a
negative value so we \emph{do not} jump, but we continue with Line
13, until we reach the second \texttt{jump} (Line 17). There we
always jump to Line 21 and execute the final \texttt{print} and \texttt{halt}
instructions. This is a total of 13 instructions executed.

\item If \texttt{left} is greater or equals to \texttt{right}, the programs will start
at Line 7 but jump Line 18 when it reaches the first
\texttt{jump} instruction (the \texttt{ACC} register would be positive or
null). The machine then continues from there until the end. This
gives us a total of 11 instructions executed.
\end{enumerate}

From this we can conclude that the first case is the worst, although
the difference is not much.

\subsection{Implicit Conversion to RAM assembly}
\label{sec:orgb15cbae}

Making explicit RAM assembly programs \textbf{is not only impractical, but
also captures specific compilation choices}. The results we got in the
previous section are valid, \emph{provided our RAM assembly is the one
yielded by the compiler}. Unfortunately there may be other ones and so
different results are possibles.

To circumvent that, we can estimate this number of instruction by find
the places in our Java programs that requires a RAM
instructions. These are:
\begin{itemize}
\item assignments, which translate into a \texttt{STORE} instruction. For
example \texttt{int i = 0} in Java. We can also consider a 
\texttt{return} statement as a form of assignment.
\item arithemtic operations, such as \texttt{+}, \texttt{-}, \texttt{*}, \texttt{/}, etc. The minimal
RAM model only provides \texttt{ADD} and \texttt{SUBTRACT} but one often assumes
additional instructions for other operations.
\item comparison operators, such as \texttt{\textasciitilde{}==}, >\textasciitilde{}, \texttt{!=}, etc. in Java. These
translate to at least one \texttt{JUMP}. As for the aritmetic we can also
assumes additional instructions that jumps on other
conditions.
\item Logical operators, such as \texttt{\&\&}, \texttt{||} or \texttt{!} in Java. They would
also translate into at least one JUMP instructions.
\end{itemize}

With this approach, we simply count the occurrence of such
statements. In our Java function (see Figure \ref{org1bd9571}), we found 1
comparison and 2 assignment (return), as follows:

\begin{enumerate}
\item If \texttt{left} is strictly less than \texttt{right}, the program does 1
comparison and 1 assignment. So a total of 2 instructions.

\item If \texttt{left} is greater of equal to \texttt{right}, then the machine carries
out 1 comparison and 1 assignment. So a total of 2 instructions.
\end{enumerate}

What we see here is that, if we abstract away the technical details of
the compilation, the two scenario are the same in this
example. Further, because the two scenario are equivalent, \emph{the average
complexity is also the same}, because in all case, we compute 1
comparison and 1 assignment.


\section{Rock Paper Scissor}
\label{sec:org7ffcc4a}

Let's look at the "Rock Paper Scissor" program, which we started
implementing during our lab session on unit testing.

It is a Java function that takes as input two integer values
representing the choices made by the two players. The body of this
function is a long conditional statement that selects the winning player
depending on the players' choice. For instance, if Player 1
chooses \texttt{ROCK} and Player 2 \texttt{PAPER}, then it returns the value 2 to
indicate that Player 2 won. When both players choose the same valid value,
the function returns the value 0 to indicate a draw. If any
player chooses anything else but \texttt{ROCK}, \texttt{PAPER} or \texttt{SCISSOR} (say
for instance \(-8\)), the function throws an exception.

\begin{verbatim}
 1  public static int play(int player1, int player2) {
 2      if (player1 == ROCK && player2 == ROCK) {
 3          return 0;
 4      } else if (player1 == ROCK && player2 == PAPER) {
 5          return 2;
 6      } else if (player1 == ROCK && player2 == SCISSOR) {
 7          return 1;
 8      } else if (player1 == PAPER && player2 == ROCK) {
 9          return 1;
10      } else if (player1 == PAPER && player2 == PAPER) {
11          return 0;
12      } else if (player1 == PAPER && player2 == SCISSOR) {
13          return 2;
14      } else if (player1 == SCISSOR && player2 == ROCK) {
15          return 1;
16      } else if (player1 == SCISSOR && player2 == PAPER) {
17          return 1;
18      } else if (player1 == SCISSOR && player2 == SCISSOR) {
19          return 0;
20      } else {
21          throw new RuntimeException("Invalid Input");
22      }
23  }
\end{verbatim}

Note the use of \texttt{return} statements within the conditional. They
interrupt the normal flow of execution and directly terminate the
function. No further instruction is executed. The same holds for the
\texttt{throw} statement, no further instruction is executed.

In this exercise, we will look at the \textbf{runtime efficiency}.

\subsection{Worst Case Analysis}
\label{sec:orgcdf080d}

So what is the worst possible scenario with respect to the runtime
efficiency. The worse possible scenario results from choosing inputs
(or unknown) that maximize the "execution path", that is, it maximise
the number of instruction executed by the machine.

In this case, the inputs that maximizes the execution path are these
inputs that trigger an exception. For example, if Player 1 chooses
\texttt{ROCK} but Player 2 chooses an invalid number, say \(1243\). This is
because every single condition will be evaluated before the function
throws an exception. Note that while every condition is evaluated, all
the return statements will be skipped since none of these conditions holds
if any value given by the players is invalid.

\textbf{How much time will this worse case scenario take?}

For the sake of simplicity, I assume a RAM machine with an extended
instruction set including comparison operators (\texttt{==}, \texttt{>}, \texttt{<=}, etc.)
and logical operators (\texttt{\&\&}, \texttt{||}, and \texttt{!}). I also assume that each
of these new operations \emph{takes 1 unit of time} to complete. Finally, I
assume a \texttt{return} statement takes 1 unit of time as well.

To find how long it take in the worse case scenario, we must find
which Java instructions will be executed, and for each, how long it
would take for a RAM machine to execute a similar job.

Let us look first at one single condition like \texttt{player1 == ROCK \&\&
player2 == ROCK} (all conditions follow this very pattern). Such an
expression contains two comparisons and one logical test, each taking
1 unit of time to execute. So a single condition takes \(3 \times 1 =
3\) units of time.

Now, to reach the line where our function throws an exception, we must
evaluate each of the nine conditions. That gives \(3 \times 9 = 27\)
units of time, to which we can add one more to account for the throw
statement. All together, \textbf{the worst case takes 28 units of time}.

\subsection{Best case Analysis}
\label{sec:orgdee2328}

By contrast with the worst case, the best case occurs for inputs that
trigger the shortest possible execution path. This occurs when the
first condition is true, that is, when both player choose \texttt{ROCK}. This
is the first case our function checks, and since it works, the
function immediately returns 0.

In this case, only one condition is evaluated, followed by a return
statement. That gives \textbf{a total of 4 units of time}: 3 units for the
condition and 1 for the return statement.

\subsection{Average case Analysis}
\label{sec:org2133369}

The limitation of the best and worst case analysis is that they do not
say much about what we should expect in average. In many situations,
these worse and base are rare events that are not representative of
a random execution.

The first step is to model the runtime efficiency. To to that, I would
define a function that maps the values chosen by the user to the time
it take our algorithm to complete.

\[
time(p_1, p_2) = \begin{cases}
4 & \text{if } p_1 = \text{rock} \land p2 = \text{rock} \\
7 & \text{if } p_1 = \text{rock} \land p2 = \text{paper} \\
10 & \text{if } p_1 = \text{rock} \land p2 = \text{scissor} \\
13 & \text{if } p_1 = \text{paper} \land p2 = \text{rock} \\
16 & \text{if } p_1 = \text{paper} \land p2 = \text{paper} \\
19 & \text{if } p_1 = \text{paper} \land p2 = \text{scissor} \\
22 & \text{if } p_1 = \text{scissor} \land p2 = \text{rock} \\
25 & \text{if } p_1 = \text{scissor} \land p2 = \text{paper} \\
28 & \text{if } p_1 = \text{scissor} \land p2 = \text{scissor} \\
28 & \text{otherwise} \\
\end{cases}
\]

The challenge here is that, \emph{without any further assumption}, we do know
often a given case shows up.

To estimate an average case, we must make an assumption about the
behavior of the players. How likely is it for Player 1 to choose
\texttt{ROCK}, to choose \texttt{PAPER},to choose \texttt{SCISSOR}, or to provide an
invalid value?. The very same holds for Player 2.


To do this rigorously---at least as much as I can---I replace the two
variable \(p_1\) and \(p_2\), by two \emph{random variables} named \(P_1\) and
\(P_2\), which represent the values chosen by Player 1 and Player 2,
respectively. These two random variables have the following properties
\begin{itemize}
\item \(P_1 \in C\) where \(C = \{\text{ROCK}, \text{PAPER}, \text{SCISSOR}, \text{ERROR}\}\)
\item \(P_1\) and \(P_2\) are uniformly distributed, that is \(\forall c \in C, \Pr[P_1 =
   c] = \Pr[P_2=c] = \frac{1}{4}\)
\end{itemize}

With this we can now express the average case as the expected value of
the time function:

\begin{align*}
    \text{average runtime} &= \text{Exp}[time(P_1, P_2)] \\
                           &= \sum_{(p_1, p_2) \in C^2} \Pr[P_1=p_1 \land P_2=p_2] \cdot time(p_1, p_2) \\
                           &= \sum_{(p_1, p_2) \in C^2} \Pr[P_1=p_1] \cdot \Pr[P_2=p_2] \cdot time(p_1, p_2) \\
\end{align*}

The definition of an \emph{expected value} in Probability tells us that it
equals the sum of all possible values weighted by their probability. I
do not know how to further simplify this expression, but we can solve
it by enumerating all the cases, as I do in Table \ref{tab:org30d0a32}, where I
list each all input cases, the runtime and its probability. We obtain
an average runtime execution of 21.25 unit of time.

\begin{table}[htbp]
\caption{\label{tab:org30d0a32}All the possible inputs and their probability}
\centering
\begin{tabular}{lllrr}
\hline
Player 1 &  Player 2 &  Probability &  Runtime & Product\\
\hline
rock & rock & 1/16 & 4 & 0.25\\
rock & paper & 1/16 & 7 & 0.4375\\
rock & scissor & 1/16 & 10 & 0.625\\
paper & rock & 1/16 & 13 & 0.8125\\
paper & paper & 1/16 & 16 & 1.\\
paper & scissor & 1/16 & 19 & 1.1875\\
scissor & rock & 1/16 & 22 & 1.375\\
scissor & paper & 1/16 & 25 & 1.5625\\
scissor & scissor & 1/16 & 28 & 1.75\\
\hline
invalid & rock & 1/16 & 28 & 1.75\\
invalid & paper & 1/16 & 28 & 1.75\\
invalid & scissor & 1/16 & 28 & 1.75\\
invalid & invalid & 1/16 & 28 & 1.75\\
rock & invalid & 1/16 & 28 & 1.75\\
paper & invalid & 1/16 & 28 & 1.75\\
scissor & invalid & 1/16 & 28 & 1.75\\
\hline
 & Totals & 1. &  & 21.25\\
\end{tabular}
\end{table}

You may have noted that the way we model the user is a gross
simplification. In practice, there is not four possible values, but a
large number since our program accepts integer values as
inputs. I see two alternatives:
\begin{itemize}
\item If we assume a more realistic computation model, for example where
memory cells are bounded to 32 bits values, then the probability of
an erroneous input would dominate and we would obtain a result very
close to the worse case. Note that \(\Pr[P=\text{invalid}] =
   1 - \frac{1}{2^{32}-3} = 0.99999999976\).
\item We could also adhere to the RAM model. There is thus an infinity
of erroneous cases. I not very confident I could solve this one.
\end{itemize}

Regardless, the takeaway is that the result we obtain depends on the
assumption we make, that is how we model the behavior of the user.


\section{Drawing Triangles}
\label{sec:orgc889274}

Let us look at a last example, a program that draws a triangle in
ASCII. Figure \ref{org600934d} shows one possible Java
implementation. It uses two \emph{nested} loops. The outer one iterates
through the rows to be printed, whereas the nested loop iterates
through a single line. One way to think of it that we print one \texttt{*} on
the first line, two \texttt{*} on the second lines, three \texttt{*} on the third
line, and so on and so forth.

\begin{verbatim}
 1  public class Triangle {
 2  
 3      public static void printTriangle(int height) {
 4          for(int row=0 ; row<height ; row++){
 5              for(int column=0 ; column<=row ; column++){
 6                  System.out.print("* ");  
 7              }  
 8              System.out.println();
 9          } 
10      }
11  
12      public static void main(String[] args) {  
13          printTriangle(6);
14      }  
15  }  
\end{verbatim}
\captionof{figure}{\label{org600934d}Drawing triangles on the console in Java}

Should you run this program, you would see something like:

The first step is to find the size of the problem. Here the size of
the problem is the height of the triangle. The higher the triangle
the more stars our algorithm prints.

Let us now turn to the worst, best and average cases, for a problem of
a given size, say 6 for instance. In this case, these three cases
are the same, because knowing the height of the triangle is enough
to estimate the time.

\begin{verbatim}
public static void printTriangle(int height)
{                              //  Cost Freq
    int row = 0;               //     1    1
    while (row < height) {     //     1    height
        printRow(row);         //     ?    height-1
        row = row + 1;         //     2    height-1 
    }
}

public static void printRow(int length)
{                                  //  Cost    Freq
    int column = 0;                //     1    1
    while (column <= length) {     //     1    length+2
        System.out.print("* ");    //     1    length+1
        column = column + 1;       //     2    length+1
    }                              //
    System.out.println();          //     1    1
}
\end{verbatim}
\captionof{figure}{\label{org4642b54}Extracting the inner loop as a separate function}

Let us start with the inner loop, which I have extracted as a
separate function in Figure \ref{org4642b54}.

\begin{align*}
time_I(\ell) &= (1 \cdot 1) + (1 \cdot (\ell + 2) + (1 \cdot (\ell+1)) + (2 \cdot (\ell + 1)) + (1 \cdot 1) \\
             &= 1 + \ell + 2 + \ell + 1 + 2\ell + 2 + 1 \\
             &= 4 \ell + 7 
\end{align*}

Now we can now look into the outer loop and plug this result as the
cost of the function call. We obtain:

\begin{align*}
time_O(h) &= (1 \cdot 1) + (1 \cdot h) + \left[ \sum_{\ell=0}^{h-1} \right 4 \ell + 7 \right] + ((h-1) \cdot 2) \\
          &= 3h -1 + \sum_{\ell=0}^{h-1} 4 \ell + 7 \\
          &= 3h - 1 + \sum_{\ell=0}^{h-1} 4 \ell + \sum_{\ell=0}^{h-1} 7 \\
          &= 3h - 1 + 4 \cdot \sum_{\ell=0}^{h-1} 4 \ell + \sum_{\ell =0}^{h-1} 7 \\
          &= 3h - 1 + 4 \cdot \sum_{\ell=0}^{h-1} \ell + 7h \\
          &= 3h - 1 + 4 \cdot \sum_{\ell=1}^{h} \ell + 7h \\
          &= 3h - 1 + 4 \cdot \frac{h(h+1)}{2} + 7h \\
          &= 10h - 1 + 2 \cdot h(h+1) + 7h \\
          &= 2h^2 + 19h - 1 \\ 
\end{align*}
\end{document}