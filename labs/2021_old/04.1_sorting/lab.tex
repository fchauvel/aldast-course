% Created 2021-09-13 Mon 22:38
% Intended LaTeX compiler: pdflatex
\documentclass[11pt]{article}
\usepackage[utf8]{inputenc}
\usepackage[T1]{fontenc}
\usepackage{graphicx}
\usepackage{grffile}
\usepackage{longtable}
\usepackage{wrapfig}
\usepackage{rotating}
\usepackage[normalem]{ulem}
\usepackage{amsmath}
\usepackage{textcomp}
\usepackage{amssymb}
\usepackage{capt-of}
\usepackage{hyperref}
\usepackage{minted}
\author{NTNU IDATA 2302}
\date{Sep. 2021}
\title{Selection Sort Analysis\\\medskip
\large Lab Session 4.1}
\hypersetup{
 pdfauthor={NTNU IDATA 2302},
 pdftitle={Selection Sort Analysis},
 pdfkeywords={},
 pdfsubject={},
 pdfcreator={Emacs 27.2 (Org mode 9.4.4)}, 
 pdflang={English}}
\begin{document}

\maketitle
\tableofcontents




\section{Introduction}
\label{sec:orgf5da430}

In this lab session, we will apply algorithm analysis and big-Oh
notation to derive the best, worst and average case of the \emph{selection
sort}.  As a reminder, the idea of the selection sort is to pick the
minimum of the given array, and move to swap it with the first
position. Because it is the minimum, it is, by definition, in the
first bucket of the sorted array. Now, we proceed in a similar fashion
with remaining elements: We pick the minimum of the remaining array,
and we put in the first place, until have looked at every element.

Listing \ref{org6bf4aa6} shows one possible implementation of the
selection sort in Ruby. For the sake of simplicity, I have expanded
fancy Ruby loops into some plain old \texttt{while} loops, so that every is
step is explicit.

\begin{listing}[htbp]
\begin{minted}[linenos,firstnumber=1]{ruby}
def selection_sort(array)                 
  index = 0                               
  while index < array.length-1            
     minimum = find_minimum(array, index) 
     swap(array, index, minimum)          
     index = index + 1                    
  end
end

def find_minimum(array, start)   
  minimum = start                
  index = start+1                
  while index < array.length     
    if array[index] < array[minimum] 
      minimum = index                
    end
    index = index + 1                
  end
  return minimum   
end                

def swap(array, left, right)
   tmp = array[right]       
   array[right] = array[left] 
   array[left] = tmp          
end                           
\end{minted}
\caption{\label{org6bf4aa6}A Ruby implementation of the selection sort}
\end{listing}

We will analysis this algorithm from the bottom up: We start with the
"helpers" functions and the then plug our results into the analysis of
the selection sort.

\section{The \texttt{swap} function}
\label{sec:org551510c}

The \texttt{swap} is the simplest because it does not contain any control
structures, such as loops or conditionals. We can look at it line by
line and estimate their respective cost and frequencies.

Each line is an assignment, which translates into a \texttt{STORE}
instruction on the RAM machine. So each line takes 1 unit of time to
run. Besides, as there is no loops or conditions, the code of the
\texttt{swap} simply executes sequentially: The machine simply executes
each line once.

We can thus write down our estimation of the runtime of the swap
function (denoted as \(\text{swap}_t\)), as
\begin{align*}
  \text{swap}_t &= (1 \times 1) + (1 \times 1) + (1 \times 1) \\
                &= 3 \\
\end{align*}

Note that the best case, the worst case and the average case are the
same, because the code runs sequentially.

\section{The \texttt{find\_minimum} function}
\label{sec:org1318736}

Let us look at the \texttt{find\_minimum} function, which returns the index
of the minimum. It accepts two arguments: An array where it must
find the minimum and a starting position, from which it should start
searching. First, the function takes the element at the starting
position and then iterates over the remaining items. For each item,
it compares it to value in the bucket currently considered as the
minimum, and, if the value is smaller it remember it as the "new"
minimum. Eventually it returns the index of the minimum.

Going through this function line by line, we get:

\begin{enumerate}
\item The first line \texttt{minimum = start} is a simple assignment, which
would translates into a \texttt{STORE} instruction in RAM assembly. This
instruction executes once. So its total cost is 1 unit of time.

\item The second line \texttt{index = start + 1} contains an assignment and an
addition, so it would translate into at least to RAM assembly
instructions, namely a \texttt{ADD} and a \texttt{STORE}. Besides, it runs only
once, so takes 2 units of time.

\item The third line is the loop condition \texttt{while index <
     array.length-1}. It contains a comparison and a subtraction and
would translate into at least 2 instructions, \texttt{SUBSTRACT} and
\texttt{JUMP}. It will run for every bucket from the given starting
position (excluded), until first value beyond the length of the
array that is, \(\ell-s+1\).

\item The fourth line is a condition that compares the value at the
current accepted minimum with the value at the current index.  It
contains a comparison so it cost 1 unit of time. we know that it
will be executed each time the machine enters the loop, that
\(\ell-s\) times. Its total cost is thus \(1 \times (\ell-s)\).

\item At Line 15, we reassign the \texttt{minimum} variable with the current
index. So it takes 1 unit of time. The problem is that we do not
know how many times this line will be executed, because it
depends on the result of the comparison on the previous
line. Here we have several cases:

\begin{itemize}
\item In \textbf{the worst case}, we always reach this line because the
previous test always succeed. This line would thus be executed
each time we enter the loop, that is \(\ell-s\) times.

\item In \textbf{the best case}, we never reach this line, because the
previous test always fails. This line would thus never be
executed and its total cost would be zero.

\item In \textbf{the average case}, we need to know how many times do does the
test succeed to know how many time this line is executed. To
model this, we need to make an assumption. We can formalize
this using a random variable, say \(N\), which denotes how many
times we will update this minimum variable.

\begin{itemize}
\item \(N\) can take \(l-s\) values, from \(s+1\), \(s+2\), \ldots, \(\ell-1\).

\item We can assume that each of this value has the same
probability to occurs, that \(\Pr[N=s+1] = \Pr[N=s+2] = \ldots =
         \frac{1}{\ell-s}\)
\end{itemize}

With this we can express the average case as a function of \(N\)
such as \(N \times (\ell - s)\).
\end{itemize}

\item Next, at Line 17, we increment the index by 1. This would entail
a \texttt{STORE} and an \texttt{ADD} instruction and cost at least 2 units of
time. It gets executed each time we enter the loop that is
\(\ell-s\) times. So the total cost is \(2 \times (\ell-s)\).

\item Finally, once the index exceeds the length of the array, we exit
the \texttt{while} loop and return the index where we found the minimum
(see Line 19). This costs 1 unit of time and is executed only
once.
\end{enumerate}

If we put all these costs together we can estimate the best, worst and
average case. In the best case, the body of the conditional (Line 15)
never runs, so we are left with:
\begin{align*}
   \text{best} & = 1 + 2 + [2 (\ell-s) + 2] + (\ell-s) + [2(\ell-s)] + 1 \\
               &= 5 (\ell-s) + 6 \\
\end{align*}

By contrast, in the worst case, the body of the conditional (Line 15)
is always executed, so we are left with:
\begin{align*}
   \text{worst}(\ell, s) & = 1 + 2 + [2 (\ell-s) + 2] + (\ell-s) + (\ell-s) + [2(\ell-s)] + 1 \\
               &= 6 (\ell-s) + 6 \\
\end{align*}

Finally, the average case requires a new parameter, namely the random
variable \(N\), but can be described as:
\begin{align*}
   \text{average}(\ell, s, N) & = 1 + 2 + [2 (\ell-s) + 2] + (\ell-s) + N + [2(\ell-s)] + 1 \\
                              &= 5(\ell-s) + N + 6 \\
\end{align*}

We can further simplify this expression by expressing it as the
expected value for all the possible values of \(N\):
\begin{align*}
   \text{average}(\ell, s) & = \text{Exp}[\text{average}(\ell, s, N)] \\
                           & = \sum_{n\,\in\, N} \Pr[N=n] \times \text{average}(\ell, s, n) \\
                           & = \sum_{n=0}^{\ell-s} \Pr[N=n] \times \text{average}(\ell, s, n) \\
                           & = \sum_{n=0}^{\ell-s} \frac{1}{\ell-s} \times \text{average}(\ell, s, n) \\
                           & = \frac{1}{\ell-s} \times \sum_{n=0}^{\ell-s} \text{average}(\ell, s, n) \\
                           & = \frac{1}{\ell-s} \times \sum_{n=0}^{\ell-s} 5(\ell-s) + n + 6 \\
                           & = \frac{1}{\ell-s} \times \left[ \sum_{n=0}^{\ell-s} 5(\ell-s) + \sum_{n=0}^{\ell-s} n + \sum_{n=0}^{\ell-s} 6 \right] \\
                           & = \frac{1}{\ell-s} \times \left[ 5(\ell-s)^2 + \frac{(\ell-s)(\ell-s+1)}{2} + 6(\ell-s) \right] \\
                           & = 5(\ell-s) + \frac{(\ell-s+1)}{2} + 6\\
                           &= \frac{11(\ell-s) + 13}{2}\\
\end{align*}

\section{The \texttt{sort} function}
\label{sec:org1f5fe51}

Finally, in the \texttt{sort} function, we iterate through the given
array. For each iteration, we find the index of the minimum and then,
we swap it with current index. Line by line we get:

\begin{enumerate}
\item The first line is an assignment, so it costs 1 unit of time and
run only once.

\item The second line is the exit condition of the \texttt{while} loop. It
contains an addition and a comparison so it takes two unit of
time. It runs for every item in array and another time once
\texttt{index} exceeds the length of the array. That gives us a total
cost of \(2 \times \ell\).

\item At line 4, we call the \texttt{find\_minimum} function, whose cost we
know from Section \ref{sec:org1318736}. The problem is
that the cost vary at each iteration, because we invoke this
function with different argument. We get something like:
\begin{align*}
    \text{cost} & = \text{average}(\ell, 0) + \text{average}(\ell, 1) + \ldots + \text{average}(\ell, \ell) \\
                & = \sum_{i=0}^{\ell} \text{average}(\ell, i) \\
                & = \sum_{i=0}^{\ell} \frac{11(\ell-i) + 13}{2}\\
                & = \frac{1}{2} \sum_{i=0}^{\ell} 11(\ell-i) + 13\\
                & = \frac{1}{2} \left[ \sum_{i=0}^{\ell} 11\ell + 13 - \sum_{i=0}^{\ell} i \left] \\
                & = \frac{1}{2} \left[ (\ell+1)(11\ell + 13) - \frac{\ell(\ell+1)}{2} \left] \\
                & = \frac{1}{2} \left[ 11\ell^2 + 24\ell +13 - \frac{\ell(\ell+1)}{2} \left] \\
                & = \frac{21\ell^2+47\ell+26}{2}
\end{align*}

\item At Line 5, we swap the minimum with the current value, and we
know from Section \ref{sec:org551510c} that it has a constant cost
of 3. Since we execute this for every element of our array, we
get a total cost of \(3\ell\).

\item At Line 6, we increment the \texttt{index} variable by 1, which cost 2
and occurs each time we enter the loop. We get a total cost of
\(2\ell\).
\end{enumerate}

We can express the average runtime of the selection sort by summing
all these costs as follows:
\begin{align*}
   \text{cost} & = 1 + 2\ell + \frac{21\ell^2+47\ell+26}{2} + 3\ell + 2 \ell \\
               & =\frac{21\ell^2+61\ell+28}{2} \\
\end{align*}



\begin{figure}[htbp]
\centering
\includegraphics[width=.9\linewidth]{./visualization.pdf}
\caption{\label{fig:org4c8c38d}Visualization of our average runtime efficiency model, together with its lower and upper bound.}
\end{figure}
\end{document}