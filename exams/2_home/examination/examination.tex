% Created 2021-10-23 Sat 07:44
% Intended LaTeX compiler: pdflatex
\documentclass[11pt]{article}
\usepackage[utf8]{inputenc}
\usepackage[T1]{fontenc}
\usepackage{graphicx}
\usepackage{grffile}
\usepackage{longtable}
\usepackage{wrapfig}
\usepackage{rotating}
\usepackage[normalem]{ulem}
\usepackage{amsmath}
\usepackage{textcomp}
\usepackage{amssymb}
\usepackage{capt-of}
\usepackage{hyperref}
\usepackage{minted}
\author{NTNU IDATA 2302}
\date{Week 10}
\title{Take-home Examination 2}
\hypersetup{
 pdfauthor={NTNU IDATA 2302},
 pdftitle={Take-home Examination 2},
 pdfkeywords={},
 pdfsubject={},
 pdfcreator={Emacs 27.2 (Org mode 9.4.4)}, 
 pdflang={English}}
\begin{document}

\maketitle
This home examination \textbf{does not account for your final grade}, but is
an opportunity to practice. Like the final examination will, it
contains four parts, each awarded 25 points (each question worths 5
points). To make the most of it, I suggest you consider the following:
\begin{itemize}
\item During the final examination, you will only have 4 hours to solve
the given problems. You may want to time-box this home examination
as well.
\item During the final examination, you \textbf{are allowed} to use \textbf{any}
written or online material.
\item For this to mirror the conditions of the final examination, you
have to take it individually.
\item I expect answers to be motivated. Questions that contain the word
\emph{prove} call for a formal and detailed proof. Otherwise, an
informal explanation (sound and complete of course) suffices.
\item Please turn your solution in \textbf{by Friday, Nov. 5, 2021}, if you
would like me to grade it.
\end{itemize}

These home examinations also help me find how much can be done in four
hours and which topics I fail to convey clearly. Any \textbf{feedback is
welcome}.

\section{Basic Knowledge (25 pts)}
\label{sec:orgc041c99}

\begin{enumerate}
\item Given the following RAM program. If the user provides the value 5,
will it print anything on the output device? Explain why?

\begin{minted}[linenos,firstnumber=1]{asm}
        READ     10
        LOAD      0
        ADD      10
        SUBTRACT 10
        STORE    10
        JUMP   done
        PRINT    10
done:   HALT   
\end{minted}

\item Is the following statement correct: \(O((x+y)^2) =
    O(x^2+y^2)\). Prove it.

\item Let us consider a hashtable using \emph{separate chaining}, equipped with the hash
function \(h(x) = x \mod 10\). What is the status of the table
after inserting the values 4371, 1323, 6173, 4199, 4344, 9679,
and 1989.

\item In the following binary search tree, what tree would result from
inserting 88 and then deleting 87? Explain why.

\begin{center}
\includegraphics[width=5cm]{bst.png}
\end{center}
\end{enumerate}



\begin{enumerate}
\item Is the following tree a valid \emph{maximum binary heap}? Explain why.

\begin{center}
\includegraphics[width=5cm]{heap.png}
\end{center}
\end{enumerate}


\section{Algorithm Analysis (25 pts)}
\label{sec:orge88be4d}

A palindrome is a word that can be read in both directions: from
left to right or from right to left. Radar, level, madam, or kayak
are examples of palindromes. In this section, we are interested into
algorithms that can decide whether a given word is a palindrome or
not, as shown by the following code snippet:

\begin{minted}[]{java}
boolean isPalindrome(char[] givenWord) {
   // Your logic here ...
}
\end{minted}

\begin{enumerate}
\item Write an \emph{iterative} function that decides whether the given word is
a palindrome or not.

\item What are the worst case time and space efficiencies of your algorithm. Prove it.

\item Write a \emph{recursive} algorithm that does the same.

\item What are the worst case time and space efficiencies of your algorithm. Prove it.

\item Is possible to improve the time and space efficiency of the recursive
function. How would you proceed?
\end{enumerate}


\section{Data Structure Design (25 pts)}
\label{sec:org5de7f26}

A \emph{cache} is a system that holds in main memory elements that have
already been used so that there is no need to compute them or to
read them again from a slower device. Caches often help to speed
up access to disks, to databases, or to the network.

Let us assume for instance that we need to retrieve some users'
profile using a remote service. To avoid sending too many requests
over the network, we decide to use a cache to store locally the
profiles that we have already retrieved. The following Java code
illustrates the behavior of such a cache.

\begin{minted}[linenos,firstnumber=1]{java}
public class Cache {

   private UserService remote;

   public Cache(UserService givenRemote) {
       this.remote = givenRemote;
   }

   public void findUserById(String userId) {
       var userProfile = this.search(userId);
       if (userProfile != null) {
           return userProfile;
       }
       var userProfile = this.remote.findUserById(userId);
       if (this.isFull()) {
           this.discardOne();
       }
       this.add(userProfile);
   }

   private UserProfile search(String userId) { /* ... */ }

   private void discardOne() { /* ... */ }
}
\end{minted}

A cache has however a limited capacity, and when the cache is full,
one must choose an entry to discard. The challenge when designing
a cache is to maximize the probability of finding what we need in
the cache (so called a cache "hit", as opposed to a cache "miss").

\begin{enumerate}
\item What data structure would you use to implement such a cache?

\item Which element would you choose to discard, and why? Provide an
algorithm for both the \texttt{search} and \texttt{discardOne} function.

\item What are the best case runtime efficiencies of a user when
using your cache design?

\item What are the worst case runtime efficiencies of finding a user
profile using the cache? (We can assume that the time to fetch a
remote profile through the network is \(t_r\)).

\item We decide to discard the \emph{least-recently locally accessed}
strategy. What data structure would you use to implement this
strategy and guarantee constant search and discard runtime?
\end{enumerate}


\section{Algorithm Design (25 pts)}
\label{sec:orge81f5b2}

We are looking at now the file system, that is, the directory
structure offered by the operating system. In a nutshell, a
directory can contain two kinds of elements: files or directories. A
file system thus forms a n-ary tree structure where nodes are
directories and leaves files. The purpose of this exercise is to
define an algorithm that lists the content of a given directory,
including the content of its sub directories. The following figure
shows the tree structure formed by the directories of a simple Java
project.

\begin{center}
\includegraphics[width=10cm]{directory.png}
\end{center}



In this picture, different icons denote directories and files. For
example, the root directory \texttt{project} contains two sub-directories,
namely \texttt{src} and \texttt{target}, and one file named \texttt{pom.xml}. Provided
with this tree, your algorithm should print something like:

\begin{verbatim}
project
   src
      main
         java
             MyClass.java 
         resources
      test
         java
             MyClassTest.java
         resources
   target
   pom.xml
\end{verbatim}

To avoid having to deal with a real and complicated file system API, we will assume
the following simplified interface. 

\begin{minted}[linenos,firstnumber=1]{java}
interface FileSystem {
    // Return a handler to the file
    File get(String path);

    // Check whether a file is a directory of not
    boolean isDirectory(File file);

    // Returns the content of the givenj directory or an empty list if
    // given a file.
    List<File> contentsOf(File file);

    // return the name of the file
    String getName(File file);
}
\end{minted}

\begin{enumerate}
\item Propose an algorithm that formats the content of a given directory as
shown previously. The key point here is to indent each entry so
that nested elements appear further on the right (compared to
their container).

\begin{minted}[]{java}
void list(FileSystem fileSystem, String givenPath) {
    // Your logic goes here
}
\end{minted}

\item What are the time and space efficiencies of your
algorithm? Explain your reasoning.

\item Propose a second algorithm that only lists the files (and not the
directories), but displays their name together with the path
that leads to them.

\begin{minted}[linenos,firstnumber=1]{java}
void listFilesOnly(FileSystem fileSystem, Path givenPath) {
     // Your logic gooes here
}
\end{minted}

Given the Java project previously shown, this algorithm should
display the following:

\begin{verbatim}
project/src/main/java/MyClass.java
project/src/test/java/MyClassTest.java
project/pom.xml
\end{verbatim}

\item What are the worst-case time and space efficiencies? Explain your
reasoning.

\item How would you estimate the average case runtime efficiency?
Explain your reasoning.
\end{enumerate}

\noindent\rule{\textwidth}{0.5pt}
\textbf{End of the examination}
\end{document}