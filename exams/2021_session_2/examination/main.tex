\documentclass{article}

\usepackage{calrsfs}
\usepackage{amsmath}
\usepackage{amsfonts}
\usepackage{minted}
\usepackage{tikz}
\usepackage[mode=buildnew]{standalone}

\title{Final Examination}
\author{NTNU 2302 / Algorithms and Data Structures}
\date{Session II \textbullet\ May 12, 2022}

\begin{document}
\maketitle


This is the final examination of the algorithm an data structure
course. It contains four parts, each having 5 questions worth 2.5
points. You have four hours.

You can describe algorithms using plain English and bullet points,
pseudocode, or some well-known programming syntax (i.e., C,
Python, Java, etc). Choose what you are the most comfortable with.

Note that you are \emph{not} required to figure out ``optimal''
algorithms (if any exist for these problems). Rather you have to show
consistency through your answers, by providing a working solution and
computing its complexity/efficiency properly.

Good luck.

\section{Basic Knowledge}

\begin{enumerate}
\item Can ``testing'' establish the correctness of an algorithm? Why?
\item Consider the following Java function. Explain (in words) what
  would happen (and why) if one invokes it with $n=-3$ as argument.
  \begin{minted}{java}
    int factorial (int n) {
      if (n==0) return 1;
      return n * factorial(n-1);
    }
  \end{minted}
\item What is an ``array'' . How does it differ from a linked-list?
\item Quick-sort runs in $O(n \log n)$. Is it correct to say that it
  also runs in $O(n^2)$. Explain your reasoning.
\item Consider the graph shown below on Figure~\ref{fig:graph}. In
  what order will the nodes be reached during a depth-first
  traversal, starting from Node A, and processing neighbor nodes
  in alphabetical order?
  \begin{figure}[htbp]
  \begin{center}
    \includestandalone[width=.6\textwidth]{images/graph.tikz}
  \end{center}
  \caption{Sample directed graph, with 6 nodes.}
  \label{fig:graph}
\end{figure}
\end{enumerate}


\section{Algorithm Analysis}

The ``Caesar cipher'' is an old encryption algorithm used during the
Roman empire. The idea is to \emph{shift} all letters by a fixed
number along the alphabet, either forward or backward. To encode and
decode, we must know the secret key, which is the length of this
shift. Here are a few examples assuming a 26 letters alphabet from 'a'
to 'z',


\begin{itemize}
\item The word ``abc'' gets encoded as ``bcd'' when the key/shift is
  1. The character 'a' becomes 'b' when shifted by 1, the character
  'b' becomes 'c', and, 'c' becomes 'd'.
\item The word ``zoo'' shifted by 2 gets encoded as ``bqq''. The
  second letter after 'z' is 'b' (we assume that the alphabet forms a
  circle). The second letter after 'o' is 'q'.
\item The word ``algorithm'' gets encoded as ``dojrulwkp'' if the
  key/shift is 3. The third character after 'a' is 'd', the third
  character after 'l' is 'o' and so on and so forth.
\end{itemize}


The following Java function is one possible implementation of the
Caesar cipher. It uses the ASCII character encoding, which preserves
the alphabet ordering. For example, in ASCII, 'a' is encoded as 97,
'b' as 98, 'c' as 99, etc. In Java, converting a ``char'' to an
``int'' yields an ASCII code.

\begin{minted}{java}
  static char[] caesarCipher(char[] givenWord, int shift) {
    char[] result = new char[givenWord.length];
    for(int index=0 ; index<givenWord.length ; index++) {
      int asciiCode = (int) givenWord[index];
      int encoded = asciiCode + shift;
      if (encoded > (int) 'z') {
        encoded -= 26;
      }
      result[index] = (char) encoded;
    }
    return result;
  }
\end{minted}


\paragraph{Questions}
\begin{enumerate}
\item Explain how the execution unfolds given the word ``zone'' and a
  shift of 4.
\item What is the \emph{size of the problem}, that is, what drives the
  runtime and memory consumption.
\item What is the best case scenario? What is the execution
  time efficiency/complexity. Explain your reasoning.
\item What is the worst case scenario? What is the execution
  time efficiency/complexity. Explain your reasoning.
\item What is the average case scenario? What is the execution
  time. Explain your reasoning. (Assume that every letter is equally
  probable).
\end{enumerate}


\section{Algorithm Design}
Consider linked-list as shown in Figure~\ref{fig:list}, where each
node points to the next one. As we build such list, it is possible to
build ``loops'' as shown on Figure~\ref{fig:list-loop}. We would
like to design a procedure to check whether the pointers that make up
the list forms a loop or not.

\begin{figure}[htbp]
  \begin{center}
    \includestandalone[width=.6\textwidth]{images/list.tikz}
  \end{center}
  \caption{A ``regular'' linked-list.}
  \label{fig:list}
\end{figure}

\begin{figure}[htbp]
  \begin{center}
    \includestandalone[width=.6\textwidth]{images/list-loop.tikz}
  \end{center}
  \caption{A ``invalid'' linked-list that includes a loop.}
  \label{fig:list-loop}
\end{figure}

For the sake of simplicity, we will assume the existence of the
following procedure that helps manipulate the nodes of these
lists:
\begin{itemize}
\item \mintinline{java}{Node getNext (Node n) {...}} returns the ``next''
  Node or null if it is not defined.
\end{itemize}

This task is about a procedure \mintinline{java}{boolean hasLoop (Node
  first) {...}}, which returns true only if the given list contains a
loop.

\paragraph{Questions}

\begin{enumerate}
\item Propose an ``iterative'' algorithm to detect such a loop. Feel
  free to add information to the node structure, if you feel it helps.
\item What is the time efficiency of your algorithm? Explain your reasoning.
\item What is the space efficiency of your algorithm? Explain your reasoning.
\item Convert your ``iterative'' algorithm into a ``recursive'' one.
\item Recursive algorithms leverage the ``call stack'' to store
  parameters of each active calls. What is the space efficiency of
  your recursive solution?  Explain your reasoning.
\end{enumerate}


% \section{Problem Solving}
% We would like to implement a container data-structure, which is
% defined by an \emph{absract data type} (ADT). This structure, denoted
% by $\mathbf{s}$, can contain an arbitrary number of elements of type
% $T$ $T$ can be any thing---think of a generic type in Java for
% example. $T$ does not make a difference for the rest of this
% exercise. Our ADT defines the following operations:
% \begin{enumerate}
% \item Insertion. We should be able to add an element $x$ to any given
%   structure $\mathbf{s}$, but only once. In other words, our
%   structure forbids duplicates. Formally, we could write:
%   \begin{align*}
%     \mathit{insert}: S \times T & \to S \\
%     \mathit{insert}\, (\mathbf{s}, x) &= \mathbf{s} \cup \{ x \}
%   \end{align*}
% \item Deletion. We should be able to remove an element $x$ from any given 
%   structure $\mathbf{s}$. Formally, we could specify that: 
%   \begin{align*}
%     \mathit{delete}: S \times T & \to S \\
%     \mathit{delete} \, (\mathbf{s}, x) & = \mathbf{s} - \{ x \}
%   \end{align*}
% \item Range queries. We want to find all the elements from
%   $\mathbf{s}$ that lays in between the two given elements $x$ and
%   $y$: Formally, we could write:
%   \begin{align*}
%     \mathit{between}: S \times T \times T & \to S \\
%     \mathit{between} \, (\mathbf{s}, x, y) & = \{ z \, | \, z \in \mathbf{s} \, \land \, x \leq z
%     \leq y \}
%     \end{align*}
% \end{enumerate}

% \paragraph{Questions}
% \begin{enumerate}
% \item What data-structure would you choose to implement this ADT?
% \item Propose an algorithm to implement the ``insert'' procedure.
% \item What is the time efficiency of your insertion procedure? Explain your reasoning.
% \item Propose an algorithm to evaluate ``range queries'', that is to
%   find all the elements in between two values.
% \item What is the time efficiency of your ``range queries'' procedure?
%   Explain your reasoning.
% \end{enumerate}

\end{document}
