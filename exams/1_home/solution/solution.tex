% Created 2021-10-06 Wed 20:59
% Intended LaTeX compiler: pdflatex
\documentclass[11pt]{article}
\usepackage[utf8]{inputenc}
\usepackage[T1]{fontenc}
\usepackage{graphicx}
\usepackage{grffile}
\usepackage{longtable}
\usepackage{wrapfig}
\usepackage{rotating}
\usepackage[normalem]{ulem}
\usepackage{amsmath}
\usepackage{textcomp}
\usepackage{amssymb}
\usepackage{capt-of}
\usepackage{hyperref}
\usepackage{minted}
\author{NTNU IDATA 2302}
\date{Week 4}
\title{Take-home Examination 1\\\medskip
\large Elements of Solution}
\hypersetup{
 pdfauthor={NTNU IDATA 2302},
 pdftitle={Take-home Examination 1},
 pdfkeywords={},
 pdfsubject={},
 pdfcreator={Emacs 27.2 (Org mode 9.4.4)}, 
 pdflang={English}}
\begin{document}

\maketitle
I present below some elements of what I think are valid
solutions. There may be other approaches just as correct. Just let me
know if you disagree or if anything is unclear.


\section{Basic Knowledge (25 pts)}
\label{sec:org7fc2a3e}

\subsection{Provided that the following RAM assembly program is written from memory address \texttt{0}. What value is stored at memory address \texttt{2} once the machine has stopped?}
\label{sec:org4649746}
\begin{minted}[linenos,firstnumber=1]{asm}
LOAD 2
ADD  2
ADD  2
STORE 2
HALT
\end{minted}

This program puts the value \(2\) in the \texttt{ACC} register, then adds
the value stored at address 2, which is the operation code for
\texttt{ADD}, say 6 for instance (it is just a convention). It adds
again this value to the \texttt{ACC} register and then stores it back
into the cell at address \texttt{2}, which then contains \(2+6+6=14\).


\subsection{Can a RAM assembly program modify its own instructions? Give an example.}
\label{sec:org58e8d33}

Yes, the previous program illustrates this point. This
program can runs only once, since at the end of the program, the
second instruction has an invalid operation code.

\subsection{How many \emph{execution paths} result from the following Java program?}
\label{sec:org69201a5}
\begin{minted}[linenos,firstnumber=1]{java}
static int maximum(int a, int b, int c) {
   if (a < b) {
      if (b < c) {
         return c;
      } else {
         return b;
      }
   } else {
     if (a < c) {
        return c;
     } else {
        return a;
     }
   }
}
\end{minted}

This programs includes two nested conditionals. The outer
conditional yields two execution paths (i.e., either \texttt{a < b} or
\texttt{a >= b}). The first case leads to another conditional with 2
possible paths, as does the second case. In total we are left
with 4 possible execution paths.

\subsection{Why does the insertion in an array have a linear worst case runtime efficiency?}
\label{sec:org4935067}

Inserting at an arbitrary position in an array implies to make room
for the new element. To do that, we must move each element at the
insertion point and beyond by one bucket towards the end of the
array. In the worst case, we insert "in front" (i.e., in bucket 0)
and we must move every single bucket to make space.

\subsection{Why does the insertion in a linked list also have a linear runtime efficiency in the worst case?}
\label{sec:orgdcc5d7e}

Inserting at an arbitrary position in a linked list implies to
traverse the list, from the first node to the insertion point. In
the worst case, we have to insert at the back of the list, and we
must hence traverse the whole list.

\section{Algorithm Analysis (25 pts)}
\label{sec:org7e929c7}

In this part, we will look at the worst case runtime complexity of
the insertion sort. To simplify the calculation, we will \textbf{only count
comparisons}.

The listing below shows a Java implementation of the insertion
sort. Insertion sort builds a sorted array by inserting new elements
a their proper position.

\begin{minted}[linenos,firstnumber=1]{java}
public static void insertionSort(int[] array) {
    for (int current=1 ; current<array.length ; current++) {
        int itemToInsert = array[current];

        // We select where to insert the current item
        int selectedPosition = current;
        for (int position=0 ; position < current ; position++){
            if (array[position] > itemToInsert) {
                selectedPosition = position;
                break;
            }
        }

        // We insert the current element at the selected position
        for (int index=current; index>selectedPosition; index--) {
            array[index] = array[index-1];
        }
        array[selectedPosition] = itemToInsert;
    }
}
\end{minted}

\subsection{Rewrite the code so that the inner \texttt{for} loops become two separate functions named \texttt{findInsertionPosition} and \texttt{insert}, respectively. Also expand all three \texttt{for} loops into while loop.}
\label{sec:org267c9fe}

One way to do this is shown in the Java code below. To preserve
the control flow, we replace the original \texttt{break} statement (line
10) by a \texttt{return} statement.

\begin{minted}[linenos,firstnumber=1]{java}
static void sort(int[] array) {
    int current = 1;
    while (current < array.length) {
        int selectedPosition =
            findInsertionPosition(array, current);
        insert(array, selectedPosition, current);
        current = current + 1;
    }
}

static int findInsertionPosition(int[] array, int limit) {
    int position = 0;
    while (position < limit){
        if (array[position] > array[limit]) {
            return position;
        }
        position = position + 1;
    }
    return limit;
}

static void insert(int[] array, int selectedPosition, int limit) {
    int itemToInsert = array[limit];
    int index = limit;
    while (index > selectedPosition) {
        array[index] = array[index-1];
        index = index - 1;
    }
    array[selectedPosition] = itemToInsert;
}
\end{minted}

\subsection{The \texttt{findInsertionPosition} function}
\label{sec:org14b1db3}

\subsubsection{What is the problem size, that is, what governs the runtime of this procedure?}
\label{sec:org2f795a7}

The problem size is given by \texttt{limit} parameter, which decides
how far we must look into the given array.

\subsubsection{In the worst case, how many comparisons are necessary?}
\label{sec:org211749d}

There are two comparisons in the \texttt{findInsertPosition} function:
One at Line 13 where we check whether we should exit the
while loop, and one at Line 14 where we check whether we have
found the right position to insert.

In the worst case, we have to iterate all the way to the limit
and eventually return the given limit as the insertion
position. This implies that:
\begin{itemize}
\item The while loop condition will be evaluated \(\text{limit} + 1\)
times;
\item The conditional statement (line 14) will be evaluated
\(limit\) times.
\end{itemize}

In the worst case, the total number of comparisons performed is
given by \(\text{time}_{fip}(\ell, i)=2 \ell + 1\), where \(\ell\)
denotes the given \texttt{limit}.

\subsection{The \texttt{insert} function}
\label{sec:org8801599}

\subsubsection{What is the problem size, that is, what governs the number comparisons?}
\label{sec:orgf2948f9}

In this insert function, we iterate (backward) from the current
position (the \texttt{limit} parameter) down to the
selected position. The further away are these two values, the
more comparisons we will perform.

\subsubsection{In the worst case, how many comparisons are needed?}
\label{sec:org4715d0e}

There is only one comparison in the \texttt{insert} function (line
25), which is the loop condition. It will be evaluated for
every index between the given limit and the selected position, and
once more when the index goes below the selected position. In the
worst case, the selected position points to the first bucket and
so we should shift all the elements forward. That is we get \(\ell\)
comparisons, where \(\ell\) denotes the given \texttt{limit}.

In the worst case, the total count of comparisons performed by the
\texttt{insert} function is given by \(\text{time}_{i}(\ell, i)=\ell\),
where \(\ell\) denotes the given \texttt{limit}.


\subsection{The \texttt{insertionSort} procedure}
\label{sec:org7b32164}

\subsubsection{What is the problem size?}
\label{sec:org440a10d}

We iterate across the given array so its length directly drives
how many comparisons we will perform.

\subsubsection{In the worst case, how many comparisons are needed?}
\label{sec:org0db9562}

The worst case occurs when every iteration triggers the worst case
of both  \texttt{insert} and \texttt{findInsertionPosition}.

\begin{align*}
  time &= \ell + \sum_{i=1}^{\ell} time_{fip}(\ell, i) + \sum_{i=0}^{\ell} time_{i}(\ell, i) \\
       &= \ell + \left[ \sum_{i=1}^{\ell} 2i + 1 \right] + \sum_{i=1}^{\ell} i \\
       &= \ell + \left[ \sum_{i=1}^{\ell} 2i + \sum_{i=1}^{\ell} 1 \right] + \frac{\ell\cdot(\ell+1)}{2} \\
       &= \ell + \left[ \sum_{i=1}^{\ell} 2i + \ell \right] + \frac{\ell\cdot(\ell+1)}{2} \\
       &= \ell + \left[ 2 \cdot \sum_{i=1}^{\ell} i + \ell \right] + \frac{\ell\cdot(\ell+1)}{2} \\
       &= \ell + \left[ 2 \cdot \frac{\ell\cdot(\ell+1)}{2}  + \ell \right] + \frac{\ell\cdot(\ell+1)}{2} \\
       &= 2\ell + (\ell\cdot(\ell+1))  + \frac{\ell\cdot(\ell+1)}{2} \\
       &= \frac{4\ell+3\ell\cdot(\ell+1)}{2} \\
       &= \frac{3\ell^2 + 7\ell}{2} \\
\end{align*}

This is will never happen in practice because these two worst
cases can never happen simultaneously. In the worst case, the
\texttt{findInsertPosition} (we insert at last) triggers the most
favorable case for \texttt{insert} (i.e., inserting at the end of the
array).

\subsection{What is the upper bound of the runtime of the insertion sort?}
\label{sec:org0adfa8b}

Let us assume that \(time(n) \in O(n^2)\). In the that, there must
exist a constant \(c\) and a constant \(n_0\) such that \(\forall n_0
   \geq n, f(n) \leq c \cdot n^2\). Assuming \(c=2\), we get:

\begin{align*}
     f(n) & \leq c \cdot n^2 \\
     \frac{3n^2 + 7n}{2}  & \leq 2 \cdot n^2 \\
     3n^2 + 7n  & \leq 4 \cdot n^2 \\
     0  & \leq n^2 - 7n \\
     0 & \leq n (n-7) \\
\end{align*}

Which holds for \(n \geq 7\), that is \(n_0=7\). We can thus conclude
that \(time \in O(n^2)\).


\section{Data Structure (25 pt)}
\label{sec:org0dc9e00}

In this problem we look at the \emph{Josephus problem} which goes as
follows. A group of \(n\) players (numbered from 1 to \(n\)) forms a
circle and a ball is passed from one player to next, \(k\) times. The
last player that receives the ball is excluded. The excluded player
gives the ball to her successor, the circle then closed and the game
restarts, with ball going through another \(k\) passes. The last person
in the game wins.


\subsection{In one single run of this game, how many time will we need to exclude a player? How many time will we need to pass the ball?}
\label{sec:org78a869a}

The winner is the last player that stands, which means that all
other players must have been excluded before, each one on a
separate round. In total, we will as many rounds as there are
players and exclude a total of \(n-1\) players.

In each round, in order to eliminate one player, we pass the
ball \(k\) times. So in total we will pass the ball \(k(n-1)\).

\subsection{Which data structure would you choose to simulate a run of this game? An array or a linked list? How would this help with deletion and traversal?}
\label{sec:orge9a9347}

I would use a doubly linked-list. To make the programming easier, I
would make it \emph{circular}, that is, I would make the last element
points towards the first one, just like in the circle of
players. The following picture illustrates this idea.

\begin{center}
\includegraphics[width=.9\linewidth]{josephus.png}
\end{center}

The circular list simplifies the traversal, since it avoids
handling going back to the beginning of the list. This is just an
optimization and it is not strictly necessary.

The back pointer (which makes our list a doubly linked list)
helps with the deletion. I can traverse the list just like the
ball goes around the circle of players, and when I have counted
\(k\) passes, I can simply delete---in constant time---the current
player.


\subsection{Outline an algorithm which simulates this process and outputs the identifier of the wining player.}
\label{sec:orge33ff70}

The overall idea is to mimic the behavior of the ball. The
following program iterates through the players until there is only
one left into the circle. Each \(k\) players (i.e., each \(k\) ball
exchanges), it deletes the player ID that "has the ball" and moves
the ball the next player. Since the list is circular, there is no
complicated condition to handle to restart from the beginning.

The following code snippet illustrates the main steps of the
algorithm in Java.
\begin{minted}[linenos,firstnumber=1]{java}
int josephus(int playerCount, int passCount) {
    CustomList players = new CustomList();
    for (int playerId=1 ; playerId<=playerCount ; playerId++) {
        players.insert(playerId);
    }

    Node playerWithTheBall = players.first;
    while (players.length > 1) {
        for (int counter=0 ; counter<passCount ; counter++) {
            playerWithTheBall = playerWithTheBall.next;
        }
        players.delete(playerWithTheBall);
        playerWithTheBall = playerWithTheBall.next;
    }
    return players.first.playerId;
}
\end{minted}

We manipulate the \texttt{Nodes} that make the list (see Line 7) and
not only the players' ID. By doing so, the \texttt{delete} operation takes
constant time, because a node directly points towards its successor
and its predecessor.

\subsection{What is the space-efficiency of your solution? Argue \emph{informally}.}
\label{sec:org8a72f43}

This solution requires a linked list of \(n\) nodes, where each node
contains an integer that identifies the player. Each node also
contains two pointers: One to the next node and one to the previous
node. That is a total of \(3n\) values, to which we must add a
pointer to the beginning of the list, a variable to hold the length
of the list and a 2 loop indexes. That gives something around \(3n +
   4\). Thus the memory consumption is linear with respect to the
number of players.

\subsection{What is the runtime-efficiency of your solution? Argue \emph{informally}.}
\label{sec:orgbcd8665}

This algorithm goes through the following steps:

\begin{enumerate}
\item It creates a list by inserting as many nodes as there are
players. The insertion in a doubly linked list takes constant
time, provided we have a pointer to the last element. Here we
have it since the list is circular: The last element, is the
one before the first one.

\item The second step is to iterate through this circular list and
to remove players until only one remains. This outer loop will
run \(n-1\) times. In each run, we iterate \(k\)
times (to simulate the ball going through \(k\) players). Getting
the next item in a linked list takes constant time, as does
the deletion. These two nested loops executed in \(k(n-1)\)
steps.
\end{enumerate}

All together we get \(n + k(n - 1)\), which is linear with respect
with respect to the size of the players and the number of ball
exchange needed to exclude a player. This \(\Theta(k \cdot n)\).


\section{Algorithm Design (25 pt)}
\label{sec:orga009949}

The purpose of this problem is to design algorithm that checks
whether a given algebraic expression has balanced parentheses. If
the given expression is well-balanced parentheses, the algorithm
should return \texttt{1}, otherwise it should output the index of first
invalid parenthesis.

\begin{minted}[]{java}
static int checkParentheses(char[] expression) {
   // Your logic goes here
}
\end{minted}

For the sake of simplicity, we assume the algebraic expression comes
as an array of character. We also assume that it only contains a
single type of parentheses, namely '(' or ')'.

Consider the following test cases:
\begin{itemize}
\item Checking \texttt{(a+1)/(2+c)} should return \texttt{-1} because the given input
has balanced parentheses;
\item Checking \texttt{(a+(2-c)*3} should return \textasciitilde{}0\textasciitilde{} because the first
parenthesis is never closed;
\item Checking \texttt{(a+b)*3)+c} should return \texttt{7} because the parenthesis
at index 7 has no match
\item Checking \texttt{a+b/c} should return \texttt{0} because it does not contain
any parentheses.
\end{itemize}

\subsection{Outline an algorithm that implements such a check.}
\label{sec:org71bd6c5}

I would traverse the expression character by character and count
the number of parentheses that need to be closed later on. To do
that I would initialize a variable \texttt{bracketToClose} such that each time
I find an open bracket, I would increment this variable, and each
time I find a closing bracket, I would decrement it. At any
point, if this variable becomes negative, there are too many
closing brackets. At the end, if the value is strictly positive,
we miss some closing brackets.

To return the index of the first invalid bracket, we must
distinguish between two cases:

\begin{enumerate}
\item If the invalid bracket is an extra closing bracket, then we
know it as soon as the \texttt{braketToClose} variable becomes
negative. So we simply return the position of the current
character.

\item If the invalid bracket is a missing opening bracket, then we
must remember where was the first opening bracket. There may
be several one but we only return the first one. So we need an
extra variable, which holds the position where a group of
bracket starts. Each time \texttt{braketToClose} goes back to zero,
we know the group was balanced, and we can then reset it when
a new group starts (i.e., when we found a new opening
bracket).
\end{enumerate}

The following listing outlines a possible Java implementation:

\begin{minted}[linenos,firstnumber=1]{java}
static int checkParentheses(char[] expression) {
    int bracketToClose = 0;
    int groupStart = -1;
    for(int index=0 ; index<expression.length ; index++) {
        if (expression[index] == '(') {
            bracketToClose += 1;
            if (bracketToClose == 1) {
                groupStart = index;
            }
        } else if (expression[index] == ')') {
            bracketToClose -= 1;

        } else {}
        if (bracketToClose < 0) { return index; }
    }
    if (bracketToClose > 0) {
        return groupStart;
    }
    return -1;
}
\end{minted}

\subsection{Explain why your algorithm satisfies the four test cases listed previously?}
\label{sec:org8292ca2}

\begin{enumerate}
\item The algorithm traverses the first expression
\texttt{(a+1)/(2+c)}. The first character is an open bracket, so
\texttt{bracketToClose} becomes 1 and \texttt{groupStart} becomes 0. At the
second bracket, which is a closing bracket, \texttt{bracketToClose}
becomes 0.  At the 3rd bracket, which is an opening bracket,
\texttt{bracketToClose} becomes 1 and \texttt{groupStart}
becomes 0. Finally, the last bracket set \texttt{bracketToclose} back
to zero so we return \texttt{-1}, as expected;
\item The algorithm traverses the second expression \texttt{(a+(2-c)*3} as
follows. The first opening bracket sets \texttt{bracketToClose} to 1
and \texttt{groupStart} to 0. The second bracket increments
\texttt{bracketToClose} to 2, whereas the 3rd bracket decreases it
back to \texttt{1}. At the end, the \texttt{bracketToClose} is positive so
we return \texttt{groupStart}, which is 0, as expected.
\item The algorithm traverses the third expression \texttt{(a+b)*3)+c} as
follows. The first bracket sets \texttt{bracketToClose} to 1 and
\texttt{groupStart} to 0. The second brackets restores
\texttt{bracketToClose} to zero, and the third one decrements it
further to \texttt{-1}. At that point, we break out of the loop and
return the current position, that is, 7 as expected.
\item The fourth expression does not contain any parentheses so the
none of the conditional statement is executed and we simply
return \texttt{-1} at the end, as expected.
\end{enumerate}

\subsection{What is the worst case runtime efficiency of your algorithm?}
\label{sec:org8b16299}

In the worst case, the expression is only made out of brackets, but
the expression is either balanced or not. Both cases force the
algorithm to traverse the whole expression, whereas an extra
closing bracket only interrupt the traversal. In the worst case, we
traverse the whole expression of brackets and we get runtime in
\(\Theta(n)\).

\subsection{What is the worst case space efficiency of your algorithm?}
\label{sec:orga3e0d61}

This algorithm does not allocate any memory besides the two
variable and the loop index. It therefore consume a constant amount
memory, regardless of the size of the expression to check.

\subsection{Do you think there is a better solution, regarding runtime and space?}
\label{sec:org7e993d0}

Regarding runtime, I do not think there exist a faster algorithm,
because I cannot see how to avoid checking all the characters of the
expression. I think the lower bound is \(\Omega(n)\).

Regarding space, this algorithm is already running in constant space
so there is no room for improvement.
\end{document}